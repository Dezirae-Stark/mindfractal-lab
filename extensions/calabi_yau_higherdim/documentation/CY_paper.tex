\documentclass[12pt]{article}
\usepackage{amsmath,amssymb,amsfonts}
\usepackage{geometry}
\geometry{margin=1in}

\title{Calabi-Yau Inspired Higher-Dimensional Consciousness Dynamics:\\
A Computational Framework}
\author{MindFractal Lab Contributors}
\date{\today}

\begin{document}

\maketitle

\begin{abstract}
We present a computational framework for exploring higher-dimensional complex state spaces inspired by the mathematical structure of Calabi-Yau manifolds. Our model implements a discrete-time nonlinear dynamical system in $\mathbb{C}^k$ with unitary evolution and nonlinear perturbations.

\textbf{DISCLAIMER}: This is a CONCEPTUAL MODELING TOOL, NOT a physical theory of spacetime or consciousness.
\end{abstract}

\section{Mathematical Framework}

\subsection{State Space}

The state space is $\mathbb{C}^k$ with states:
\begin{equation}
z = (z_1, z_2, \ldots, z_k) \in \mathbb{C}^k
\end{equation}

\subsection{Dynamical System}

The update rule is:
\begin{equation}
z_{n+1} = U z_n + \varepsilon (z_n \odot z_n) + c
\end{equation}

where:
\begin{itemize}
\item $U \in \mathbb{C}^{k \times k}$ is unitary ($U^\dagger U = I$)
\item $\varepsilon \in \mathbb{R}^+$ controls nonlinearity
\item $\odot$ denotes element-wise product
\item $c \in \mathbb{C}^k$ is the parameter vector
\end{itemize}

\subsection{Jacobian}

The Jacobian at state $z$ is:
\begin{equation}
J(z) = U + \varepsilon \, \text{diag}(2z_1, 2z_2, \ldots, 2z_k)
\end{equation}

\section{Parameter Space Fractals}

For fixed $(U, \varepsilon, z_0)$, define the bounded set:
\begin{equation}
\mathcal{M} = \{ c \in \mathbb{C}^k : \text{orbit} \{z_n\} \text{ remains bounded} \}
\end{equation}

This set typically has fractal boundaries.

\section{Lyapunov Exponents}

The largest Lyapunov exponent is:
\begin{equation}
\lambda = \lim_{N \to \infty} \frac{1}{N} \sum_{n=1}^N \log \| J(z_n) v_n \|
\end{equation}

where $v_n$ is a unit tangent vector, renormalized at each step.

\section{CY-Inspired Geometry}

\subsection{Hermitian Metric}

We define toy Hermitian metrics:
\begin{equation}
g_{ij}(z) = \delta_{ij} \quad \text{(flat)}
\end{equation}

or position-dependent:
\begin{equation}
g_{ij}(z) = \frac{\delta_{ij}}{1 + \|z\|^2}
\end{equation}

\subsection{Curvature Proxy}

Ricci curvature proxy:
\begin{equation}
\text{Ric}_{\text{proxy}}(z) = \log|\det J(z)|
\end{equation}

\textbf{Note}: These are heuristic diagnostics, not actual geometric curvatures.

\section{Attractor Types}

\begin{itemize}
\item \textbf{Fixed Point}: $z^* = U z^* + \varepsilon (z^* \odot z^*) + c$
\item \textbf{Periodic}: $z_{n+p} = z_n$
\item \textbf{Chaotic}: $\lambda > 0$ and bounded
\item \textbf{Unbounded}: $\|z_n\| \to \infty$
\end{itemize}

\section{Conclusions}

This framework provides a mathematically-grounded tool for exploring high-dimensional complex dynamics inspired by Calabi-Yau geometry. It is a \textbf{conceptual modeling tool}, not a physical theory.

\bibliographystyle{plain}
\begin{thebibliography}{9}
\bibitem{candelas} P. Candelas et al., \textit{Vacuum configurations for superstrings}, Nuclear Physics B 258 (1985).
\bibitem{strogatz} S.H. Strogatz, \textit{Nonlinear Dynamics and Chaos}, Westview Press (2015).
\bibitem{ott} E. Ott, \textit{Chaos in Dynamical Systems}, Cambridge (2002).
\end{thebibliography}

\end{document}
