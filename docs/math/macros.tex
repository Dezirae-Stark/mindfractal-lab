% ============================================================================
% MindFractal Lab — LaTeX Macros
% Common mathematical notation for the Fractal Consciousness Model
% ============================================================================

% ----------------------------------------------------------------------------
% Number Sets and Spaces
% ----------------------------------------------------------------------------
\newcommand{\R}{\mathbb{R}}                     % Real numbers
\newcommand{\C}{\mathbb{C}}                     % Complex numbers
\newcommand{\Z}{\mathbb{Z}}                     % Integers
\newcommand{\N}{\mathbb{N}}                     % Natural numbers

% State/parameter spaces
\newcommand{\Pcal}{\mathcal{P}}                 % Possibility manifold
\newcommand{\Bcal}{\mathcal{B}}                 % Basin of attraction
\newcommand{\Acal}{\mathcal{A}}                 % Attractor set
\newcommand{\Tcal}{\mathcal{T}}                 % Timeline space
\newcommand{\Fcal}{\mathcal{F}}                 % Function family

% ----------------------------------------------------------------------------
% Vectors and Matrices
% ----------------------------------------------------------------------------
\newcommand{\vx}{\mathbf{x}}                    % State vector (real)
\newcommand{\vz}{\mathbf{z}}                    % State vector (complex)
\newcommand{\vc}{\mathbf{c}}                    % Parameter vector
\newcommand{\veta}{\boldsymbol{\eta}}           % Noise vector
\newcommand{\vzero}{\mathbf{0}}                 % Zero vector

\newcommand{\mA}{\mathbf{A}}                    % Linear feedback matrix
\newcommand{\mB}{\mathbf{B}}                    % Nonlinear coupling matrix
\newcommand{\mW}{\mathbf{W}}                    % Weight matrix
\newcommand{\mU}{\mathbf{U}}                    % Unitary matrix
\newcommand{\mH}{\mathbf{H}}                    % Hermitian matrix
\newcommand{\mI}{\mathbf{I}}                    % Identity matrix
\newcommand{\mJ}{\mathbf{J}}                    % Jacobian matrix

% ----------------------------------------------------------------------------
% Operators and Functions
% ----------------------------------------------------------------------------
\newcommand{\diag}{\operatorname{diag}}         % Diagonal operator
\newcommand{\sech}{\operatorname{sech}}         % Hyperbolic secant
\newcommand{\orbit}{\operatorname{orbit}}       % Orbit function
\newcommand{\sgn}{\operatorname{sgn}}           % Sign function
\newcommand{\tr}{\operatorname{tr}}             % Trace
\newcommand{\rank}{\operatorname{rank}}         % Rank
\newcommand{\spec}{\operatorname{spec}}         % Spectrum

% Norms
\newcommand{\norm}[1]{\left\| #1 \right\|}      % Generic norm
\newcommand{\abs}[1]{\left| #1 \right|}         % Absolute value

% Element-wise product (Hadamard)
\newcommand{\had}{\odot}                        % Hadamard product

% ----------------------------------------------------------------------------
% Dynamical Systems Notation
% ----------------------------------------------------------------------------
\newcommand{\lyap}{\lambda}                     % Lyapunov exponent
\newcommand{\lyapmax}{\lambda_{\max}}           % Maximum Lyapunov exponent
\newcommand{\lyapspec}{\boldsymbol{\lambda}}    % Lyapunov spectrum

% Fixed point notation
\newcommand{\xstar}{\vx^*}                      % Fixed point (real)
\newcommand{\zstar}{\vz^*}                      % Fixed point (complex)

% Time indices
\newcommand{\tn}{_{n}}                          % Time n subscript
\newcommand{\tnp}{_{n+1}}                       % Time n+1 subscript

% ----------------------------------------------------------------------------
% Calabi-Yau Extension Notation
% ----------------------------------------------------------------------------
\newcommand{\CY}{\text{CY}}                     % Calabi-Yau abbreviation
\newcommand{\eps}{\varepsilon}                  % Epsilon for nonlinearity

% Update rules
\newcommand{\Ftanh}{F_{\tanh}}                  % Tanh update rule
\newcommand{\Fsigmoid}{F_{\sigma}}              % Sigmoid update rule
\newcommand{\Fthree}{F_{3D}}                    % 3D update rule
\newcommand{\Fcy}{F_{\CY}}                      % CY update rule

% ----------------------------------------------------------------------------
% Metric and Distance Notation
% ----------------------------------------------------------------------------
\newcommand{\dP}{d_{\Pcal}}                     % Distance on possibility manifold
\newcommand{\dH}{d_H}                           % Hausdorff distance
\newcommand{\dimH}{\dim_H}                      % Hausdorff dimension
\newcommand{\dimB}{\dim_B}                      % Box-counting dimension

% ----------------------------------------------------------------------------
% Common Expressions
% ----------------------------------------------------------------------------
% State update equation
\newcommand{\stateupdate}{\vx\tnp = \mA\vx\tn + \mB\tanh(\mW\vx\tn) + \vc}

% Complex state update
\newcommand{\cstateupdate}{\vz\tnp = \mU\vz\tn + \eps(\vz\tn \had \vz\tn) + \vc}

% Jacobian expression
\newcommand{\jacobian}{\mJ(\vx) = \mA + \mB \cdot \diag\left(\sech^2(\mW\vx)\right) \cdot \mW}

% Lyapunov exponent limit
\newcommand{\lyaplimit}{\lyap = \lim_{n \to \infty} \frac{1}{n} \sum_{k=0}^{n-1} \log \norm{\mJ(\vx_k)}}

% Possibility manifold definition
\newcommand{\Pdef}{\Pcal = \left\{ (\vz_0, \vc, F) : \orbit(\vz_0, \vc, F) \text{ bounded} \right\}}

% ----------------------------------------------------------------------------
% Psychological Trait Mapping
% ----------------------------------------------------------------------------
\newcommand{\traitO}{O}                         % Openness
\newcommand{\traitV}{V}                         % Volatility
\newcommand{\traitI}{I}                         % Integration
\newcommand{\traitF}{F}                         % Focus

% ----------------------------------------------------------------------------
% Theorem Environments (requires amsthm)
% ----------------------------------------------------------------------------
% \newtheorem{theorem}{Theorem}[section]
% \newtheorem{lemma}[theorem]{Lemma}
% \newtheorem{proposition}[theorem]{Proposition}
% \newtheorem{corollary}[theorem]{Corollary}
% \theoremstyle{definition}
% \newtheorem{definition}[theorem]{Definition}
% \newtheorem{example}[theorem]{Example}
% \theoremstyle{remark}
% \newtheorem{remark}[theorem]{Remark}

% ============================================================================
% End of Macros
% ============================================================================
