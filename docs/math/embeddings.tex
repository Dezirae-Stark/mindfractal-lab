% ============================================================================
% Embeddings — Latent Space Representations
% MindFractal Lab Mathematical Documentation
% ============================================================================

\documentclass[11pt, a4paper]{article}

\usepackage{amsmath, amssymb, amsthm}
\usepackage{mathtools}
\usepackage{physics}
\usepackage{hyperref}

% ============================================================================
% MindFractal Lab — LaTeX Macros
% Common mathematical notation for the Fractal Consciousness Model
% ============================================================================

% ----------------------------------------------------------------------------
% Number Sets and Spaces
% ----------------------------------------------------------------------------
\newcommand{\R}{\mathbb{R}}                     % Real numbers
\newcommand{\C}{\mathbb{C}}                     % Complex numbers
\newcommand{\Z}{\mathbb{Z}}                     % Integers
\newcommand{\N}{\mathbb{N}}                     % Natural numbers

% State/parameter spaces
\newcommand{\Pcal}{\mathcal{P}}                 % Possibility manifold
\newcommand{\Bcal}{\mathcal{B}}                 % Basin of attraction
\newcommand{\Acal}{\mathcal{A}}                 % Attractor set
\newcommand{\Tcal}{\mathcal{T}}                 % Timeline space
\newcommand{\Fcal}{\mathcal{F}}                 % Function family

% ----------------------------------------------------------------------------
% Vectors and Matrices
% ----------------------------------------------------------------------------
\newcommand{\vx}{\mathbf{x}}                    % State vector (real)
\newcommand{\vz}{\mathbf{z}}                    % State vector (complex)
\newcommand{\vc}{\mathbf{c}}                    % Parameter vector
\newcommand{\veta}{\boldsymbol{\eta}}           % Noise vector
\newcommand{\vzero}{\mathbf{0}}                 % Zero vector

\newcommand{\mA}{\mathbf{A}}                    % Linear feedback matrix
\newcommand{\mB}{\mathbf{B}}                    % Nonlinear coupling matrix
\newcommand{\mW}{\mathbf{W}}                    % Weight matrix
\newcommand{\mU}{\mathbf{U}}                    % Unitary matrix
\newcommand{\mH}{\mathbf{H}}                    % Hermitian matrix
\newcommand{\mI}{\mathbf{I}}                    % Identity matrix
\newcommand{\mJ}{\mathbf{J}}                    % Jacobian matrix

% ----------------------------------------------------------------------------
% Operators and Functions
% ----------------------------------------------------------------------------
\newcommand{\diag}{\operatorname{diag}}         % Diagonal operator
\newcommand{\sech}{\operatorname{sech}}         % Hyperbolic secant
\newcommand{\orbit}{\operatorname{orbit}}       % Orbit function
\newcommand{\sgn}{\operatorname{sgn}}           % Sign function
\newcommand{\tr}{\operatorname{tr}}             % Trace
\newcommand{\rank}{\operatorname{rank}}         % Rank
\newcommand{\spec}{\operatorname{spec}}         % Spectrum

% Norms
\newcommand{\norm}[1]{\left\| #1 \right\|}      % Generic norm
\newcommand{\abs}[1]{\left| #1 \right|}         % Absolute value

% Element-wise product (Hadamard)
\newcommand{\had}{\odot}                        % Hadamard product

% ----------------------------------------------------------------------------
% Dynamical Systems Notation
% ----------------------------------------------------------------------------
\newcommand{\lyap}{\lambda}                     % Lyapunov exponent
\newcommand{\lyapmax}{\lambda_{\max}}           % Maximum Lyapunov exponent
\newcommand{\lyapspec}{\boldsymbol{\lambda}}    % Lyapunov spectrum

% Fixed point notation
\newcommand{\xstar}{\vx^*}                      % Fixed point (real)
\newcommand{\zstar}{\vz^*}                      % Fixed point (complex)

% Time indices
\newcommand{\tn}{_{n}}                          % Time n subscript
\newcommand{\tnp}{_{n+1}}                       % Time n+1 subscript

% ----------------------------------------------------------------------------
% Calabi-Yau Extension Notation
% ----------------------------------------------------------------------------
\newcommand{\CY}{\text{CY}}                     % Calabi-Yau abbreviation
\newcommand{\eps}{\varepsilon}                  % Epsilon for nonlinearity

% Update rules
\newcommand{\Ftanh}{F_{\tanh}}                  % Tanh update rule
\newcommand{\Fsigmoid}{F_{\sigma}}              % Sigmoid update rule
\newcommand{\Fthree}{F_{3D}}                    % 3D update rule
\newcommand{\Fcy}{F_{\CY}}                      % CY update rule

% ----------------------------------------------------------------------------
% Metric and Distance Notation
% ----------------------------------------------------------------------------
\newcommand{\dP}{d_{\Pcal}}                     % Distance on possibility manifold
\newcommand{\dH}{d_H}                           % Hausdorff distance
\newcommand{\dimH}{\dim_H}                      % Hausdorff dimension
\newcommand{\dimB}{\dim_B}                      % Box-counting dimension

% ----------------------------------------------------------------------------
% Common Expressions
% ----------------------------------------------------------------------------
% State update equation
\newcommand{\stateupdate}{\vx\tnp = \mA\vx\tn + \mB\tanh(\mW\vx\tn) + \vc}

% Complex state update
\newcommand{\cstateupdate}{\vz\tnp = \mU\vz\tn + \eps(\vz\tn \had \vz\tn) + \vc}

% Jacobian expression
\newcommand{\jacobian}{\mJ(\vx) = \mA + \mB \cdot \diag\left(\sech^2(\mW\vx)\right) \cdot \mW}

% Lyapunov exponent limit
\newcommand{\lyaplimit}{\lyap = \lim_{n \to \infty} \frac{1}{n} \sum_{k=0}^{n-1} \log \norm{\mJ(\vx_k)}}

% Possibility manifold definition
\newcommand{\Pdef}{\Pcal = \left\{ (\vz_0, \vc, F) : \orbit(\vz_0, \vc, F) \text{ bounded} \right\}}

% ----------------------------------------------------------------------------
% Psychological Trait Mapping
% ----------------------------------------------------------------------------
\newcommand{\traitO}{O}                         % Openness
\newcommand{\traitV}{V}                         % Volatility
\newcommand{\traitI}{I}                         % Integration
\newcommand{\traitF}{F}                         % Focus

% ----------------------------------------------------------------------------
% Theorem Environments (requires amsthm)
% ----------------------------------------------------------------------------
% \newtheorem{theorem}{Theorem}[section]
% \newtheorem{lemma}[theorem]{Lemma}
% \newtheorem{proposition}[theorem]{Proposition}
% \newtheorem{corollary}[theorem]{Corollary}
% \theoremstyle{definition}
% \newtheorem{definition}[theorem]{Definition}
% \newtheorem{example}[theorem]{Example}
% \theoremstyle{remark}
% \newtheorem{remark}[theorem]{Remark}

% ============================================================================
% End of Macros
% ============================================================================


\title{Embeddings and Latent Spaces}
\author{MindFractal Lab}
\date{\today}

\newtheorem{theorem}{Theorem}[section]
\newtheorem{lemma}[theorem]{Lemma}
\newtheorem{proposition}[theorem]{Proposition}
\newtheorem{corollary}[theorem]{Corollary}
\theoremstyle{definition}
\newtheorem{definition}[theorem]{Definition}
\newtheorem{example}[theorem]{Example}
\theoremstyle{remark}
\newtheorem{remark}[theorem]{Remark}

\begin{document}

\maketitle

\begin{abstract}
We formalize embedding maps from the Possibility Manifold $\Pcal$ and dynamical orbits into lower-dimensional latent spaces. These embeddings enable machine learning analysis, dimensionality reduction, and classification of dynamical behaviors. We develop both linear and nonlinear embedding frameworks.
\end{abstract}

\tableofcontents

% ----------------------------------------------------------------------------
\section{Introduction}
% ----------------------------------------------------------------------------

The high-dimensional spaces of fractal dynamics require dimensionality reduction for:
\begin{itemize}
    \item \textbf{Visualization}: Project to 2D/3D for human interpretation
    \item \textbf{Classification}: Group similar dynamical behaviors
    \item \textbf{Interpolation}: Navigate between configurations
    \item \textbf{Generative modeling}: Sample new configurations
\end{itemize}

% ----------------------------------------------------------------------------
\section{Embedding Framework}
% ----------------------------------------------------------------------------

\subsection{Basic Definitions}

\begin{definition}[Embedding Map]
An embedding is a map $\Phi: \Pcal \to \R^m$ (or $\C^m$) that preserves relevant structure from the Possibility Manifold.
\end{definition}

\begin{definition}[Latent Space]
The latent space $\mathcal{L} = \Phi(\Pcal) \subseteq \R^m$ is the image of $\Pcal$ under embedding $\Phi$.
\end{definition}

\subsection{Desirable Properties}

An ideal embedding satisfies:
\begin{enumerate}
    \item \textbf{Injectivity}: $\Phi(p_1) = \Phi(p_2) \Rightarrow p_1 = p_2$ (or approximate)
    \item \textbf{Continuity}: Small changes in $p$ yield small changes in $\Phi(p)$
    \item \textbf{Structure preservation}: Nearby points in $\Pcal$ remain nearby in $\mathcal{L}$
    \item \textbf{Interpretability}: Dimensions in $\mathcal{L}$ have meaningful interpretations
\end{enumerate}

% ----------------------------------------------------------------------------
\section{Orbit-Based Embeddings}
% ----------------------------------------------------------------------------

\subsection{Trajectory Summary Statistics}

\begin{definition}[Statistical Embedding]
Map orbit to summary statistics:
\begin{equation}
    \Phi_{\text{stat}}(p) = \left(\bar{\vz}, \text{Var}(\vz), \lyap, E, \ldots\right)
    \label{eq:stat_embed}
\end{equation}
where:
\begin{itemize}
    \item $\bar{\vz} = \frac{1}{N}\sum_{k=0}^{N-1} \vz_k$ is the orbit mean
    \item $\text{Var}(\vz)$ is component-wise variance
    \item $\lyap$ is the Lyapunov exponent
    \item $E$ is energy or other dynamical invariant
\end{itemize}
\end{definition}

\subsection{Time-Delay Embedding}

\begin{definition}[Takens Embedding]
For scalar observable $h: \C^n \to \R$:
\begin{equation}
    \Phi_{\text{delay}}(p) = \left(h(\vz_0), h(\vz_\tau), h(\vz_{2\tau}), \ldots, h(\vz_{(d-1)\tau})\right)
    \label{eq:takens}
\end{equation}
with delay $\tau$ and embedding dimension $d$.
\end{definition}

\begin{theorem}[Takens' Theorem]
For generic $h$ and sufficiently large $d \geq 2\dim(\Acal) + 1$, the delay embedding is a diffeomorphism onto its image.
\end{theorem}

\subsection{Recurrence-Based Embedding}

\begin{definition}[Recurrence Quantification]
Compute recurrence matrix and extract features:
\begin{equation}
    R_{ij} = \Theta(\epsilon - \norm{\vz_i - \vz_j})
\end{equation}
where $\Theta$ is the Heaviside function. The embedding uses recurrence rate, determinism, entropy, etc.
\end{definition}

% ----------------------------------------------------------------------------
\section{Configuration Embeddings}
% ----------------------------------------------------------------------------

\subsection{Direct Parameter Embedding}

\begin{definition}[Parameter Flattening]
Flatten configuration to vector:
\begin{equation}
    \Phi_{\text{flat}}(p) = \left(\Re(\vz_0), \Im(\vz_0), \Re(\vc), \Im(\vc), \text{index}(F)\right)
    \label{eq:flat}
\end{equation}
\end{definition}

This provides a $4n + 1$ dimensional representation for $n$-dimensional complex dynamics.

\subsection{Feature Engineering}

\begin{definition}[Engineered Features]
Domain-specific features:
\begin{equation}
    \Phi_{\text{eng}}(p) = \left(\norm{\vz_0}, \arg(z_{0,1}), \norm{\vc}, \text{stability}(\vc), \ldots\right)
\end{equation}
\end{definition}

% ----------------------------------------------------------------------------
\section{Learned Embeddings}
% ----------------------------------------------------------------------------

\subsection{Autoencoder Framework}

\begin{definition}[Autoencoder]
An autoencoder consists of:
\begin{align}
    \text{Encoder}: & \quad \Phi: \Pcal \to \R^m \\
    \text{Decoder}: & \quad \Psi: \R^m \to \Pcal
\end{align}
trained to minimize reconstruction loss:
\begin{equation}
    \mathcal{L} = \mathbb{E}_{p \sim \Pcal}\left[\norm{p - \Psi(\Phi(p))}^2\right]
    \label{eq:ae_loss}
\end{equation}
\end{definition}

\subsection{Variational Autoencoder}

\begin{definition}[VAE Embedding]
The VAE encoder outputs distribution parameters:
\begin{equation}
    \Phi_{\text{VAE}}(p) = (\boldsymbol{\mu}(p), \boldsymbol{\sigma}(p))
\end{equation}
with latent $\vz \sim \mathcal{N}(\boldsymbol{\mu}, \diag(\boldsymbol{\sigma}^2))$.

The loss includes reconstruction and KL divergence:
\begin{equation}
    \mathcal{L}_{\text{VAE}} = \mathcal{L}_{\text{recon}} + \beta \cdot D_{\text{KL}}\left(q(\vz|p) \| p(\vz)\right)
\end{equation}
\end{definition}

\subsection{Contrastive Learning}

\begin{definition}[Contrastive Embedding]
Train embedding such that:
\begin{itemize}
    \item Similar configurations have similar embeddings
    \item Dissimilar configurations have distant embeddings
\end{itemize}

Loss function (e.g., InfoNCE):
\begin{equation}
    \mathcal{L}_{\text{NCE}} = -\log \frac{\exp(\Phi(p)^\top \Phi(p^+) / \tau)}{\sum_{p^-} \exp(\Phi(p)^\top \Phi(p^-) / \tau)}
    \label{eq:nce}
\end{equation}
where $p^+$ is a positive pair and $p^-$ are negatives.
\end{definition}

% ----------------------------------------------------------------------------
\section{Dimensionality Reduction}
% ----------------------------------------------------------------------------

\subsection{Principal Component Analysis}

\begin{definition}[PCA Embedding]
For data matrix $\mX \in \R^{N \times d}$ with samples as rows:
\begin{equation}
    \Phi_{\text{PCA}}(\vx) = \mV_m^\top (\vx - \bar{\vx})
\end{equation}
where $\mV_m$ contains top $m$ eigenvectors of covariance matrix $\mC = \frac{1}{N}\mX^\top\mX$.
\end{definition}

\subsection{t-SNE}

\begin{definition}[t-SNE Embedding]
Minimize KL divergence between high-dimensional affinities $p_{ij}$ and low-dimensional affinities $q_{ij}$:
\begin{equation}
    \mathcal{L}_{\text{t-SNE}} = \sum_{i \neq j} p_{ij} \log \frac{p_{ij}}{q_{ij}}
\end{equation}
where $q_{ij}$ uses Student's t-distribution in the embedding space.
\end{definition}

\subsection{UMAP}

\begin{definition}[UMAP Embedding]
Based on fuzzy topological representation:
\begin{equation}
    \mathcal{L}_{\text{UMAP}} = \sum_{i,j} \left[p_{ij} \log\frac{p_{ij}}{q_{ij}} + (1-p_{ij}) \log\frac{1-p_{ij}}{1-q_{ij}}\right]
\end{equation}
Preserves both local and some global structure.
\end{definition}

% ----------------------------------------------------------------------------
\section{Spectral Embeddings}
% ----------------------------------------------------------------------------

\subsection{Laplacian Eigenmaps}

\begin{definition}[Graph Laplacian Embedding]
Construct similarity graph with adjacency $W_{ij}$ and degree matrix $D$. Compute eigenvectors of:
\begin{equation}
    L\mathbf{v} = \lambda D\mathbf{v}
\end{equation}
where $L = D - W$ is the graph Laplacian. Embedding uses smallest nonzero eigenvectors.
\end{definition}

\subsection{Diffusion Maps}

\begin{definition}[Diffusion Embedding]
Define diffusion operator $P = D^{-1}W$ and compute eigenvectors:
\begin{equation}
    \Phi_{\text{diff}}(p) = (\lambda_1^t \psi_1(p), \lambda_2^t \psi_2(p), \ldots, \lambda_m^t \psi_m(p))
\end{equation}
where $t$ is the diffusion time scale.
\end{definition}

% ----------------------------------------------------------------------------
\section{Koopman Embeddings}
% ----------------------------------------------------------------------------

\subsection{Koopman Operator}

\begin{definition}[Koopman Operator]
For dynamics $\vz\tnp = F(\vz\tn)$, the Koopman operator $\mathcal{K}$ acts on observables:
\begin{equation}
    (\mathcal{K}g)(\vz) = g(F(\vz))
\end{equation}
\end{definition}

\begin{theorem}[Koopman Spectral Decomposition]
For suitable observables:
\begin{equation}
    g(\vz\tn) = \sum_{j} \langle g, \phi_j \rangle \lambda_j^n \phi_j(\vz_0)
\end{equation}
where $\phi_j$ are Koopman eigenfunctions with eigenvalues $\lambda_j$.
\end{definition}

\subsection{Dynamic Mode Decomposition}

\begin{definition}[DMD Embedding]
From trajectory data, construct:
\begin{equation}
    \mX' \approx \mA \mX
\end{equation}
where columns of $\mX$ are states and $\mX'$ are next states. The DMD modes and eigenvalues provide an embedding:
\begin{equation}
    \Phi_{\text{DMD}}(\vz) = (\mathbf{v}_1^\top \vz, \mathbf{v}_2^\top \vz, \ldots, \mathbf{v}_m^\top \vz)
\end{equation}
where $\mathbf{v}_j$ are DMD modes.
\end{definition}

% ----------------------------------------------------------------------------
\section{Classification in Latent Space}
% ----------------------------------------------------------------------------

\subsection{Stability Classification}

\begin{definition}[Stability Classifier]
A classifier $C: \R^m \to \{0, 1, 2, 3\}$ trained to predict:
\begin{equation}
    C(\Phi(p)) = \begin{cases}
        0 & \text{stable fixed point} \\
        1 & \text{limit cycle} \\
        2 & \text{chaotic} \\
        3 & \text{divergent}
    \end{cases}
\end{equation}
\end{definition}

\subsection{Attractor Type Prediction}

\begin{definition}[Attractor Classifier]
Multi-class classification of attractor types based on orbit features:
\begin{equation}
    C_{\text{att}}: \Phi(\orbit(p)) \to \{\text{fixed}, \text{cycle}, \text{torus}, \text{strange}\}
\end{equation}
\end{definition}

\subsection{Boundary Detection}

\begin{definition}[Boundary Classifier]
Binary classifier for basin boundary proximity:
\begin{equation}
    C_{\text{boundary}}(\Phi(p)) = \begin{cases}
        1 & \text{if } d(p, \partial\Pcal) < \epsilon \\
        0 & \text{otherwise}
    \end{cases}
\end{equation}
\end{definition}

% ----------------------------------------------------------------------------
\section{Generative Models}
% ----------------------------------------------------------------------------

\subsection{Latent Space Sampling}

\begin{definition}[Configuration Generator]
Sample new configurations via:
\begin{enumerate}
    \item Sample $\vz \sim p(\vz)$ from latent distribution
    \item Decode: $\hat{p} = \Psi(\vz)$
    \item Validate: Check $\hat{p} \in \Pcal$
\end{enumerate}
\end{definition}

\subsection{Interpolation}

\begin{definition}[Latent Interpolation]
To interpolate between $p_1, p_2 \in \Pcal$:
\begin{equation}
    p(t) = \Psi\left((1-t)\Phi(p_1) + t\Phi(p_2)\right)
\end{equation}
\end{definition}

\begin{remark}
Latent interpolation often produces smoother transitions than direct configuration interpolation, especially near bifurcations.
\end{remark}

% ----------------------------------------------------------------------------
\section{Metric Preservation}
% ----------------------------------------------------------------------------

\subsection{Distortion Measures}

\begin{definition}[Embedding Distortion]
The distortion of embedding $\Phi$ is:
\begin{equation}
    \text{distortion}(\Phi) = \max_{p_1, p_2} \left|\frac{\norm{\Phi(p_1) - \Phi(p_2)}}{\dP(p_1, p_2)} - 1\right|
\end{equation}
\end{definition}

\subsection{Stress Function}

\begin{definition}[MDS Stress]
The stress function measures embedding quality:
\begin{equation}
    \text{stress} = \sqrt{\frac{\sum_{i<j}(d_{ij} - \delta_{ij})^2}{\sum_{i<j} d_{ij}^2}}
\end{equation}
where $d_{ij} = \dP(p_i, p_j)$ and $\delta_{ij} = \norm{\Phi(p_i) - \Phi(p_j)}$.
\end{definition}

% ----------------------------------------------------------------------------
\section{Applications}
% ----------------------------------------------------------------------------

\subsection{Parameter Space Exploration}

Use embeddings to:
\begin{itemize}
    \item Cluster similar parameter regimes
    \item Identify regions of stability/chaos
    \item Navigate between dynamical behaviors
\end{itemize}

\subsection{Model Selection}

\begin{itemize}
    \item Compare different update rules in shared latent space
    \item Identify which model best fits observed data
    \item Transfer knowledge between model families
\end{itemize}

\subsection{Anomaly Detection}

\begin{itemize}
    \item Detect configurations far from typical behavior
    \item Identify transitions between regimes
    \item Flag numerical instabilities
\end{itemize}

\end{document}
