% ============================================================================
% Visualization Algorithms
% MindFractal Lab Mathematical Documentation
% ============================================================================

\documentclass[11pt, a4paper]{article}

\usepackage{amsmath, amssymb, amsthm}
\usepackage{mathtools}
\usepackage{physics}
\usepackage{hyperref}
\usepackage{algorithm}
\usepackage{algorithmic}

% ============================================================================
% MindFractal Lab — LaTeX Macros
% Common mathematical notation for the Fractal Consciousness Model
% ============================================================================

% ----------------------------------------------------------------------------
% Number Sets and Spaces
% ----------------------------------------------------------------------------
\newcommand{\R}{\mathbb{R}}                     % Real numbers
\newcommand{\C}{\mathbb{C}}                     % Complex numbers
\newcommand{\Z}{\mathbb{Z}}                     % Integers
\newcommand{\N}{\mathbb{N}}                     % Natural numbers

% State/parameter spaces
\newcommand{\Pcal}{\mathcal{P}}                 % Possibility manifold
\newcommand{\Bcal}{\mathcal{B}}                 % Basin of attraction
\newcommand{\Acal}{\mathcal{A}}                 % Attractor set
\newcommand{\Tcal}{\mathcal{T}}                 % Timeline space
\newcommand{\Fcal}{\mathcal{F}}                 % Function family

% ----------------------------------------------------------------------------
% Vectors and Matrices
% ----------------------------------------------------------------------------
\newcommand{\vx}{\mathbf{x}}                    % State vector (real)
\newcommand{\vz}{\mathbf{z}}                    % State vector (complex)
\newcommand{\vc}{\mathbf{c}}                    % Parameter vector
\newcommand{\veta}{\boldsymbol{\eta}}           % Noise vector
\newcommand{\vzero}{\mathbf{0}}                 % Zero vector

\newcommand{\mA}{\mathbf{A}}                    % Linear feedback matrix
\newcommand{\mB}{\mathbf{B}}                    % Nonlinear coupling matrix
\newcommand{\mW}{\mathbf{W}}                    % Weight matrix
\newcommand{\mU}{\mathbf{U}}                    % Unitary matrix
\newcommand{\mH}{\mathbf{H}}                    % Hermitian matrix
\newcommand{\mI}{\mathbf{I}}                    % Identity matrix
\newcommand{\mJ}{\mathbf{J}}                    % Jacobian matrix

% ----------------------------------------------------------------------------
% Operators and Functions
% ----------------------------------------------------------------------------
\newcommand{\diag}{\operatorname{diag}}         % Diagonal operator
\newcommand{\sech}{\operatorname{sech}}         % Hyperbolic secant
\newcommand{\orbit}{\operatorname{orbit}}       % Orbit function
\newcommand{\sgn}{\operatorname{sgn}}           % Sign function
\newcommand{\tr}{\operatorname{tr}}             % Trace
\newcommand{\rank}{\operatorname{rank}}         % Rank
\newcommand{\spec}{\operatorname{spec}}         % Spectrum

% Norms
\newcommand{\norm}[1]{\left\| #1 \right\|}      % Generic norm
\newcommand{\abs}[1]{\left| #1 \right|}         % Absolute value

% Element-wise product (Hadamard)
\newcommand{\had}{\odot}                        % Hadamard product

% ----------------------------------------------------------------------------
% Dynamical Systems Notation
% ----------------------------------------------------------------------------
\newcommand{\lyap}{\lambda}                     % Lyapunov exponent
\newcommand{\lyapmax}{\lambda_{\max}}           % Maximum Lyapunov exponent
\newcommand{\lyapspec}{\boldsymbol{\lambda}}    % Lyapunov spectrum

% Fixed point notation
\newcommand{\xstar}{\vx^*}                      % Fixed point (real)
\newcommand{\zstar}{\vz^*}                      % Fixed point (complex)

% Time indices
\newcommand{\tn}{_{n}}                          % Time n subscript
\newcommand{\tnp}{_{n+1}}                       % Time n+1 subscript

% ----------------------------------------------------------------------------
% Calabi-Yau Extension Notation
% ----------------------------------------------------------------------------
\newcommand{\CY}{\text{CY}}                     % Calabi-Yau abbreviation
\newcommand{\eps}{\varepsilon}                  % Epsilon for nonlinearity

% Update rules
\newcommand{\Ftanh}{F_{\tanh}}                  % Tanh update rule
\newcommand{\Fsigmoid}{F_{\sigma}}              % Sigmoid update rule
\newcommand{\Fthree}{F_{3D}}                    % 3D update rule
\newcommand{\Fcy}{F_{\CY}}                      % CY update rule

% ----------------------------------------------------------------------------
% Metric and Distance Notation
% ----------------------------------------------------------------------------
\newcommand{\dP}{d_{\Pcal}}                     % Distance on possibility manifold
\newcommand{\dH}{d_H}                           % Hausdorff distance
\newcommand{\dimH}{\dim_H}                      % Hausdorff dimension
\newcommand{\dimB}{\dim_B}                      % Box-counting dimension

% ----------------------------------------------------------------------------
% Common Expressions
% ----------------------------------------------------------------------------
% State update equation
\newcommand{\stateupdate}{\vx\tnp = \mA\vx\tn + \mB\tanh(\mW\vx\tn) + \vc}

% Complex state update
\newcommand{\cstateupdate}{\vz\tnp = \mU\vz\tn + \eps(\vz\tn \had \vz\tn) + \vc}

% Jacobian expression
\newcommand{\jacobian}{\mJ(\vx) = \mA + \mB \cdot \diag\left(\sech^2(\mW\vx)\right) \cdot \mW}

% Lyapunov exponent limit
\newcommand{\lyaplimit}{\lyap = \lim_{n \to \infty} \frac{1}{n} \sum_{k=0}^{n-1} \log \norm{\mJ(\vx_k)}}

% Possibility manifold definition
\newcommand{\Pdef}{\Pcal = \left\{ (\vz_0, \vc, F) : \orbit(\vz_0, \vc, F) \text{ bounded} \right\}}

% ----------------------------------------------------------------------------
% Psychological Trait Mapping
% ----------------------------------------------------------------------------
\newcommand{\traitO}{O}                         % Openness
\newcommand{\traitV}{V}                         % Volatility
\newcommand{\traitI}{I}                         % Integration
\newcommand{\traitF}{F}                         % Focus

% ----------------------------------------------------------------------------
% Theorem Environments (requires amsthm)
% ----------------------------------------------------------------------------
% \newtheorem{theorem}{Theorem}[section]
% \newtheorem{lemma}[theorem]{Lemma}
% \newtheorem{proposition}[theorem]{Proposition}
% \newtheorem{corollary}[theorem]{Corollary}
% \theoremstyle{definition}
% \newtheorem{definition}[theorem]{Definition}
% \newtheorem{example}[theorem]{Example}
% \theoremstyle{remark}
% \newtheorem{remark}[theorem]{Remark}

% ============================================================================
% End of Macros
% ============================================================================


\title{Visualization Algorithms for Fractal Dynamics}
\author{MindFractal Lab}
\date{\today}

\newtheorem{theorem}{Theorem}[section]
\newtheorem{lemma}[theorem]{Lemma}
\newtheorem{proposition}[theorem]{Proposition}
\theoremstyle{definition}
\newtheorem{definition}[theorem]{Definition}
\newtheorem{example}[theorem]{Example}
\theoremstyle{remark}
\newtheorem{remark}[theorem]{Remark}

\begin{document}

\maketitle

\begin{abstract}
We present algorithms for visualizing fractal dynamical systems, including fractal slices, Calabi-Yau projections, possibility manifold navigation, and parameter space exploration. These algorithms enable interactive exploration of high-dimensional dynamics through carefully designed 2D and 3D projections.
\end{abstract}

\tableofcontents

% ----------------------------------------------------------------------------
\section{Overview}
% ----------------------------------------------------------------------------

Visualization of fractal dynamics presents unique challenges:
\begin{itemize}
    \item High-dimensional state and parameter spaces
    \item Fractal boundary structure requiring high resolution
    \item Dynamic behavior requiring trajectory animation
    \item Multi-scale structure requiring zoom capability
\end{itemize}

We address these through specialized algorithms for different visualization tasks.

% ----------------------------------------------------------------------------
\section{Phase Portrait Visualization}
% ----------------------------------------------------------------------------

\subsection{Single Trajectory}

\begin{algorithm}[H]
\caption{Plot Single Trajectory}
\begin{algorithmic}[1]
\REQUIRE Initial state $\vz_0$, parameters $(\mA, \mB, \mW, \vc)$, iterations $N$
\ENSURE 2D trajectory plot

\STATE $\vz \leftarrow \vz_0$
\STATE trajectory $\leftarrow [\vz_0]$
\FOR{$n = 1$ to $N$}
    \STATE $\vz \leftarrow \mA\vz + \mB\tanh(\mW\vz) + \vc$
    \STATE trajectory.append($\vz$)
\ENDFOR
\STATE Plot trajectory[:, 0] vs trajectory[:, 1] with color gradient by time
\end{algorithmic}
\end{algorithm}

\subsection{Phase Space Vector Field}

\begin{algorithm}[H]
\caption{Vector Field Visualization}
\begin{algorithmic}[1]
\REQUIRE Grid bounds $[-L, L]^2$, resolution $n$, parameters
\ENSURE Vector field plot

\STATE Create grid $(x_i, y_j)$ for $i, j \in \{0, \ldots, n-1\}$
\FOR{each grid point $\vz = (x_i, y_j)$}
    \STATE $\vz' \leftarrow f(\vz) - \vz$ \COMMENT{Displacement vector}
    \STATE Normalize: $\hat{\vz}' \leftarrow \vz' / \max(1, \norm{\vz'})$
\ENDFOR
\STATE Plot arrows at grid points with direction $\hat{\vz}'$
\end{algorithmic}
\end{algorithm}

% ----------------------------------------------------------------------------
\section{Basin of Attraction}
% ----------------------------------------------------------------------------

\subsection{Attractor Classification Grid}

\begin{algorithm}[H]
\caption{Basin of Attraction Map}
\begin{algorithmic}[1]
\REQUIRE Grid bounds $[-L, L]^2$, resolution $n$, max iterations $N_{\max}$
\ENSURE Color-coded basin map

\STATE Create empty $n \times n$ classification array
\STATE Identify fixed points $\{\vz^*_k\}$ via Newton's method
\FOR{$i = 0$ to $n-1$}
    \FOR{$j = 0$ to $n-1$}
        \STATE $\vz_0 \leftarrow (-L + 2L \cdot i/n, -L + 2L \cdot j/n)$
        \STATE $\vz \leftarrow \vz_0$
        \FOR{$m = 1$ to $N_{\max}$}
            \STATE $\vz \leftarrow f(\vz)$
            \IF{$\norm{\vz} > R_{\text{escape}}$}
                \STATE classification[$i, j$] $\leftarrow -1$ (escaped)
                \STATE \textbf{break}
            \ENDIF
            \FOR{each fixed point $\vz^*_k$}
                \IF{$\norm{\vz - \vz^*_k} < \epsilon$}
                    \STATE classification[$i, j$] $\leftarrow k$
                    \STATE \textbf{break}
                \ENDIF
            \ENDFOR
        \ENDFOR
    \ENDFOR
\ENDFOR
\STATE Assign colors to classification values
\STATE Display as heatmap
\end{algorithmic}
\end{algorithm}

\subsection{Boundary Detection}

\begin{algorithm}[H]
\caption{Fractal Boundary Detection}
\begin{algorithmic}[1]
\REQUIRE Basin classification array $C$
\ENSURE Boundary mask

\STATE boundary $\leftarrow$ zeros\_like($C$)
\FOR{$i = 1$ to $n-2$}
    \FOR{$j = 1$ to $n-2$}
        \STATE neighbors $\leftarrow \{C[i\pm1, j], C[i, j\pm1]\}$
        \IF{any neighbor $\neq C[i, j]$}
            \STATE boundary[$i, j$] $\leftarrow 1$
        \ENDIF
    \ENDFOR
\ENDFOR
\RETURN boundary
\end{algorithmic}
\end{algorithm}

% ----------------------------------------------------------------------------
\section{Lyapunov Exponent Maps}
% ----------------------------------------------------------------------------

\subsection{Parameter-Space Lyapunov Map}

\begin{algorithm}[H]
\caption{Lyapunov Parameter Map}
\begin{algorithmic}[1]
\REQUIRE Parameter bounds $[-L_c, L_c]^2$, resolution $n$, fixed $\vz_0$
\ENSURE Lyapunov exponent heatmap

\STATE Create $n \times n$ array $\Lambda$
\FOR{$i = 0$ to $n-1$}
    \FOR{$j = 0$ to $n-1$}
        \STATE $c_1 \leftarrow -L_c + 2L_c \cdot i/n$
        \STATE $c_2 \leftarrow -L_c + 2L_c \cdot j/n$
        \STATE $\vc \leftarrow (c_1, c_2)$
        \STATE $\Lambda[i, j] \leftarrow$ ComputeLyapunov($\vz_0, \vc$)
    \ENDFOR
\ENDFOR
\STATE Apply colormap: blue (stable) $\to$ white (neutral) $\to$ red (chaotic)
\end{algorithmic}
\end{algorithm}

\subsection{Lyapunov Computation Subroutine}

\begin{algorithm}[H]
\caption{ComputeLyapunov}
\begin{algorithmic}[1]
\REQUIRE Initial state $\vz_0$, parameters, iterations $N$, transient $N_{\text{trans}}$
\ENSURE Largest Lyapunov exponent $\lambda$

\STATE $\vz \leftarrow \vz_0$
\STATE $\mathbf{v} \leftarrow$ random unit vector
\STATE $S \leftarrow 0$

\COMMENT{Transient phase}
\FOR{$k = 1$ to $N_{\text{trans}}$}
    \STATE $\vz \leftarrow f(\vz)$
\ENDFOR

\COMMENT{Lyapunov accumulation}
\FOR{$k = 1$ to $N$}
    \STATE $\mJ \leftarrow \mA + \mB \cdot \diag(\sech^2(\mW\vz)) \cdot \mW$
    \STATE $\mathbf{v} \leftarrow \mJ \cdot \mathbf{v}$
    \STATE $S \leftarrow S + \log\norm{\mathbf{v}}$
    \STATE $\mathbf{v} \leftarrow \mathbf{v} / \norm{\mathbf{v}}$
    \STATE $\vz \leftarrow f(\vz)$
\ENDFOR

\RETURN $\lambda = S / N$
\end{algorithmic}
\end{algorithm}

% ----------------------------------------------------------------------------
\section{Fractal Slice Visualization}
% ----------------------------------------------------------------------------

\subsection{Mandelbrot-Style Slice}

\begin{algorithm}[H]
\caption{Mandelbrot-Style Parameter Slice}
\begin{algorithmic}[1]
\REQUIRE Complex parameter bounds, resolution $n$, max iterations $N_{\max}$
\ENSURE Escape-time colored fractal

\STATE Create $n \times n$ array escape\_time
\FOR{$i = 0$ to $n-1$}
    \FOR{$j = 0$ to $n-1$}
        \STATE $c \leftarrow$ complex(bounds, $i$, $j$)
        \STATE $\vc \leftarrow$ embed\_parameter($c$)
        \STATE $\vz \leftarrow \vz_0$
        \FOR{$m = 1$ to $N_{\max}$}
            \STATE $\vz \leftarrow f(\vz; \vc)$
            \IF{$\norm{\vz} > R_{\text{escape}}$}
                \STATE escape\_time[$i, j$] $\leftarrow m$
                \STATE \textbf{break}
            \ENDIF
        \ENDFOR
        \IF{not escaped}
            \STATE escape\_time[$i, j$] $\leftarrow N_{\max}$
        \ENDIF
    \ENDFOR
\ENDFOR
\STATE Apply smooth coloring: color($m$) = palette($m + 1 - \log_2(\log_2(\norm{\vz}))$)
\end{algorithmic}
\end{algorithm}

\subsection{Julia-Style Slice}

\begin{algorithm}[H]
\caption{Julia Set Slice}
\begin{algorithmic}[1]
\REQUIRE Fixed parameter $\vc$, initial condition bounds, resolution $n$
\ENSURE Julia set visualization

\COMMENT{Similar to Mandelbrot but varying $\vz_0$ instead of $\vc$}
\FOR{each grid point $(x, y)$}
    \STATE $\vz_0 \leftarrow (x, y)$
    \STATE Iterate with fixed $\vc$
    \STATE Color by escape time or attractor reached
\ENDFOR
\end{algorithmic}
\end{algorithm}

% ----------------------------------------------------------------------------
\section{CY Projection Algorithms}
% ----------------------------------------------------------------------------

\subsection{Complex to Real Projection}

\begin{definition}[CY Projection Methods]
For $\vz \in \C^k$, define projections to $\R^2$:
\begin{align}
    P_{\text{Re-Im}}(\vz) &= (\Re(z_1), \Im(z_1)) \\
    P_{\text{mag-phase}}(\vz) &= (\abs{z_1}, \arg(z_1)) \\
    P_{\text{stereo}}(\vz) &= \text{stereographic}(z_1 / z_2)
\end{align}
\end{definition}

\begin{algorithm}[H]
\caption{CY Slice Viewer}
\begin{algorithmic}[1]
\REQUIRE CY dynamics parameters, slice specification
\ENSURE 2D visualization of CY dynamics

\STATE Choose projection $P: \C^k \to \R^2$
\STATE Fix all but 2 complex dimensions
\FOR{each point in 2D slice}
    \STATE Embed into $\C^k$
    \STATE Iterate CY dynamics
    \STATE Classify: bounded/escaped/attractor type
    \STATE Color accordingly
\ENDFOR
\end{algorithmic}
\end{algorithm}

\subsection{Multi-Slice Animation}

\begin{algorithm}[H]
\caption{CY Multi-Slice Animation}
\begin{algorithmic}[1]
\REQUIRE Slice parameter range $[s_{\min}, s_{\max}]$, number of frames $N_f$
\ENSURE Animation of slice evolution

\FOR{$k = 0$ to $N_f - 1$}
    \STATE $s \leftarrow s_{\min} + (s_{\max} - s_{\min}) \cdot k / (N_f - 1)$
    \STATE Compute slice at parameter $s$
    \STATE Render frame $k$
\ENDFOR
\STATE Compile frames into animation
\end{algorithmic}
\end{algorithm}

% ----------------------------------------------------------------------------
\section{Possibility Manifold Navigator}
% ----------------------------------------------------------------------------

\subsection{Manifold Exploration Interface}

\begin{algorithm}[H]
\caption{Possibility Navigator}
\begin{algorithmic}[1]
\REQUIRE Current configuration $p = (\vz_0, \vc, F)$
\ENSURE Interactive exploration interface

\LOOP
    \STATE Display current orbit visualization
    \STATE Show Lyapunov exponent and stability classification
    \STATE Show position in parameter space

    \STATE Read user input: $\Delta\vz_0$, $\Delta\vc$, or $\Delta F$
    \STATE $p' \leftarrow p + (\Delta\vz_0, \Delta\vc, \Delta F)$

    \IF{$p' \in \Pcal$}
        \STATE $p \leftarrow p'$
        \STATE Animate transition from old to new orbit
    \ELSE
        \STATE Display warning: configuration escapes
    \ENDIF
\ENDLOOP
\end{algorithmic}
\end{algorithm}

\subsection{Timeline Visualization}

\begin{algorithm}[H]
\caption{Timeline Path Visualization}
\begin{algorithmic}[1]
\REQUIRE Start configuration $p_0$, end configuration $p_1$, steps $N$
\ENSURE Visualization of timeline $\gamma: p_0 \to p_1$

\FOR{$k = 0$ to $N$}
    \STATE $t \leftarrow k / N$
    \STATE $p(t) \leftarrow (1-t)p_0 + t p_1$
    \STATE Compute orbit from $p(t)$
    \STATE Plot projected orbit in subplot $k$
    \STATE Record Lyapunov exponent
\ENDFOR
\STATE Plot Lyapunov exponent vs $t$
\STATE Highlight bifurcations where $\lambda$ changes sign
\end{algorithmic}
\end{algorithm}

\subsection{Branching Visualization}

\begin{algorithm}[H]
\caption{Bifurcation Branch Visualization}
\begin{algorithmic}[1]
\REQUIRE Bifurcation point $p^*$, perturbation directions $\{\mathbf{v}_k\}$
\ENSURE Visualization of branching structure

\STATE Plot orbit at $p^*$
\FOR{each direction $\mathbf{v}_k$}
    \FOR{$\epsilon \in \{-\delta, -\delta/2, \delta/2, \delta\}$}
        \STATE $p' \leftarrow p^* + \epsilon \mathbf{v}_k$
        \STATE Compute and plot orbit from $p'$ (semi-transparent)
    \ENDFOR
\ENDFOR
\STATE Color-code by attractor reached
\end{algorithmic}
\end{algorithm}

% ----------------------------------------------------------------------------
\section{3D Visualization}
% ----------------------------------------------------------------------------

\subsection{3D Attractor Rendering}

\begin{algorithm}[H]
\caption{3D Attractor Visualization}
\begin{algorithmic}[1]
\REQUIRE 3D dynamics model, initial state $\vz_0$, iterations $N$
\ENSURE 3D trajectory plot

\STATE trajectory $\leftarrow$ []
\STATE $\vz \leftarrow \vz_0$
\FOR{$n = 1$ to $N$}
    \STATE $\vz \leftarrow f(\vz)$
    \STATE trajectory.append($\vz$)
\ENDFOR
\STATE Plot 3D line with color gradient by iteration
\STATE Add rotation animation for interactive viewing
\end{algorithmic}
\end{algorithm}

\subsection{3D Basin Slices}

\begin{algorithm}[H]
\caption{3D Basin Visualization}
\begin{algorithmic}[1]
\REQUIRE 3D state bounds, resolution $n$, slice positions
\ENSURE Stacked 2D slices or isosurface

\COMMENT{Option 1: Stacked slices}
\FOR{$z$ in slice\_positions}
    \STATE Compute 2D basin at $z = \text{const}$
    \STATE Render as textured plane at height $z$
\ENDFOR

\COMMENT{Option 2: Isosurface}
\STATE Compute 3D Lyapunov field on grid
\STATE Extract isosurface at $\lambda = 0$
\STATE Render using marching cubes
\end{algorithmic}
\end{algorithm}

% ----------------------------------------------------------------------------
\section{Animation Algorithms}
% ----------------------------------------------------------------------------

\subsection{Trajectory Animation}

\begin{algorithm}[H]
\caption{Animated Trajectory}
\begin{algorithmic}[1]
\REQUIRE Trajectory data, background (basin map), frame rate
\ENSURE Animated GIF/video

\STATE Render background image
\FOR{frame $k = 1$ to $N_{\text{frames}}$}
    \STATE $i_{\text{end}} \leftarrow k \cdot N / N_{\text{frames}}$
    \STATE Plot trajectory[0:$i_{\text{end}}$] on background
    \STATE Mark current point with symbol
    \STATE Save frame
\ENDFOR
\STATE Compile frames to animation
\end{algorithmic}
\end{algorithm}

\subsection{Parameter Sweep Animation}

\begin{algorithm}[H]
\caption{Parameter Sweep Animation}
\begin{algorithmic}[1]
\REQUIRE Parameter path $\vc(t)$ for $t \in [0, 1]$, frames $N_f$
\ENSURE Animation showing dynamics evolution

\FOR{$k = 0$ to $N_f - 1$}
    \STATE $t \leftarrow k / (N_f - 1)$
    \STATE $\vc \leftarrow \vc(t)$
    \STATE Compute representative orbit
    \STATE Render phase portrait
    \STATE Overlay parameter value indicator
    \STATE Save frame
\ENDFOR
\end{algorithmic}
\end{algorithm}

% ----------------------------------------------------------------------------
\section{Coloring Schemes}
% ----------------------------------------------------------------------------

\subsection{Escape Time Coloring}

\begin{definition}[Smooth Escape Time]
For escape at iteration $n$ with final state $\vz_n$:
\begin{equation}
    \nu = n + 1 - \frac{\log(\log\norm{\vz_n})}{\log 2}
\end{equation}
Map $\nu$ to color via palette.
\end{definition}

\subsection{Lyapunov Coloring}

\begin{definition}[Lyapunov Color Map]
Map Lyapunov exponent to color:
\begin{equation}
    \text{color}(\lambda) = \begin{cases}
        \text{blue}(|\lambda|) & \lambda < 0 \\
        \text{white} & \lambda \approx 0 \\
        \text{red}(\lambda) & \lambda > 0
    \end{cases}
\end{equation}
\end{definition}

\subsection{Orbit Trap Coloring}

\begin{definition}[Orbit Trap]
Color by minimum distance to geometric shape $S$:
\begin{equation}
    d_{\min} = \min_{k} d(\vz_k, S)
\end{equation}
Common shapes: circles, lines, crosses, grids.
\end{definition}

% ----------------------------------------------------------------------------
\section{Performance Optimization}
% ----------------------------------------------------------------------------

\subsection{GPU Parallelization}

Key parallelization strategies:
\begin{itemize}
    \item Each pixel computed independently (embarrassingly parallel)
    \item Use GPU shaders for real-time rendering
    \item Batch trajectory computations across parameter grid
\end{itemize}

\subsection{Progressive Rendering}

\begin{algorithm}[H]
\caption{Progressive Refinement}
\begin{algorithmic}[1]
\REQUIRE Target resolution $n$
\ENSURE Progressively refined image

\FOR{level $L = 2, 4, 8, \ldots, n$}
    \STATE Compute at resolution $L \times L$
    \STATE Display upscaled result
    \STATE Allow user interaction (zoom, pan)
\ENDFOR
\end{algorithmic}
\end{algorithm}

\subsection{Caching}

\begin{itemize}
    \item Cache fixed point locations
    \item Cache Lyapunov exponents for parameter grid
    \item Use memoization for repeated computations
\end{itemize}

% ----------------------------------------------------------------------------
\section{WebGL/WebGPU Implementation}
% ----------------------------------------------------------------------------

\subsection{Shader-Based Rendering}

\begin{definition}[Fragment Shader Structure]
For web-based visualization:
\begin{verbatim}
uniform vec2 c;          // Parameter
uniform mat2 A, B, W;    // System matrices
varying vec2 uv;         // Pixel coordinates

void main() {
    vec2 z = uv * scale + offset;
    int escape = 0;
    for (int i = 0; i < MAX_ITER; i++) {
        z = A * z + B * tanh(W * z) + c;
        if (length(z) > ESCAPE_R) {
            escape = i;
            break;
        }
    }
    gl_FragColor = palette(escape);
}
\end{verbatim}
\end{definition}

\subsection{Interactive Controls}

\begin{itemize}
    \item Mouse drag: pan view
    \item Scroll: zoom in/out
    \item Click: show orbit from point
    \item Sliders: adjust parameters in real-time
\end{itemize}

\end{document}
