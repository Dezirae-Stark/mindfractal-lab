% ============================================================================
% Base Model — 2D/3D Real Fractal Dynamics
% MindFractal Lab Mathematical Documentation
% ============================================================================

\documentclass[11pt, a4paper]{article}

\usepackage{amsmath, amssymb, amsthm}
\usepackage{mathtools}
\usepackage{physics}
\usepackage{hyperref}

% ============================================================================
% MindFractal Lab — LaTeX Macros
% Common mathematical notation for the Fractal Consciousness Model
% ============================================================================

% ----------------------------------------------------------------------------
% Number Sets and Spaces
% ----------------------------------------------------------------------------
\newcommand{\R}{\mathbb{R}}                     % Real numbers
\newcommand{\C}{\mathbb{C}}                     % Complex numbers
\newcommand{\Z}{\mathbb{Z}}                     % Integers
\newcommand{\N}{\mathbb{N}}                     % Natural numbers

% State/parameter spaces
\newcommand{\Pcal}{\mathcal{P}}                 % Possibility manifold
\newcommand{\Bcal}{\mathcal{B}}                 % Basin of attraction
\newcommand{\Acal}{\mathcal{A}}                 % Attractor set
\newcommand{\Tcal}{\mathcal{T}}                 % Timeline space
\newcommand{\Fcal}{\mathcal{F}}                 % Function family

% ----------------------------------------------------------------------------
% Vectors and Matrices
% ----------------------------------------------------------------------------
\newcommand{\vx}{\mathbf{x}}                    % State vector (real)
\newcommand{\vz}{\mathbf{z}}                    % State vector (complex)
\newcommand{\vc}{\mathbf{c}}                    % Parameter vector
\newcommand{\veta}{\boldsymbol{\eta}}           % Noise vector
\newcommand{\vzero}{\mathbf{0}}                 % Zero vector

\newcommand{\mA}{\mathbf{A}}                    % Linear feedback matrix
\newcommand{\mB}{\mathbf{B}}                    % Nonlinear coupling matrix
\newcommand{\mW}{\mathbf{W}}                    % Weight matrix
\newcommand{\mU}{\mathbf{U}}                    % Unitary matrix
\newcommand{\mH}{\mathbf{H}}                    % Hermitian matrix
\newcommand{\mI}{\mathbf{I}}                    % Identity matrix
\newcommand{\mJ}{\mathbf{J}}                    % Jacobian matrix

% ----------------------------------------------------------------------------
% Operators and Functions
% ----------------------------------------------------------------------------
\newcommand{\diag}{\operatorname{diag}}         % Diagonal operator
\newcommand{\sech}{\operatorname{sech}}         % Hyperbolic secant
\newcommand{\orbit}{\operatorname{orbit}}       % Orbit function
\newcommand{\sgn}{\operatorname{sgn}}           % Sign function
\newcommand{\tr}{\operatorname{tr}}             % Trace
\newcommand{\rank}{\operatorname{rank}}         % Rank
\newcommand{\spec}{\operatorname{spec}}         % Spectrum

% Norms
\newcommand{\norm}[1]{\left\| #1 \right\|}      % Generic norm
\newcommand{\abs}[1]{\left| #1 \right|}         % Absolute value

% Element-wise product (Hadamard)
\newcommand{\had}{\odot}                        % Hadamard product

% ----------------------------------------------------------------------------
% Dynamical Systems Notation
% ----------------------------------------------------------------------------
\newcommand{\lyap}{\lambda}                     % Lyapunov exponent
\newcommand{\lyapmax}{\lambda_{\max}}           % Maximum Lyapunov exponent
\newcommand{\lyapspec}{\boldsymbol{\lambda}}    % Lyapunov spectrum

% Fixed point notation
\newcommand{\xstar}{\vx^*}                      % Fixed point (real)
\newcommand{\zstar}{\vz^*}                      % Fixed point (complex)

% Time indices
\newcommand{\tn}{_{n}}                          % Time n subscript
\newcommand{\tnp}{_{n+1}}                       % Time n+1 subscript

% ----------------------------------------------------------------------------
% Calabi-Yau Extension Notation
% ----------------------------------------------------------------------------
\newcommand{\CY}{\text{CY}}                     % Calabi-Yau abbreviation
\newcommand{\eps}{\varepsilon}                  % Epsilon for nonlinearity

% Update rules
\newcommand{\Ftanh}{F_{\tanh}}                  % Tanh update rule
\newcommand{\Fsigmoid}{F_{\sigma}}              % Sigmoid update rule
\newcommand{\Fthree}{F_{3D}}                    % 3D update rule
\newcommand{\Fcy}{F_{\CY}}                      % CY update rule

% ----------------------------------------------------------------------------
% Metric and Distance Notation
% ----------------------------------------------------------------------------
\newcommand{\dP}{d_{\Pcal}}                     % Distance on possibility manifold
\newcommand{\dH}{d_H}                           % Hausdorff distance
\newcommand{\dimH}{\dim_H}                      % Hausdorff dimension
\newcommand{\dimB}{\dim_B}                      % Box-counting dimension

% ----------------------------------------------------------------------------
% Common Expressions
% ----------------------------------------------------------------------------
% State update equation
\newcommand{\stateupdate}{\vx\tnp = \mA\vx\tn + \mB\tanh(\mW\vx\tn) + \vc}

% Complex state update
\newcommand{\cstateupdate}{\vz\tnp = \mU\vz\tn + \eps(\vz\tn \had \vz\tn) + \vc}

% Jacobian expression
\newcommand{\jacobian}{\mJ(\vx) = \mA + \mB \cdot \diag\left(\sech^2(\mW\vx)\right) \cdot \mW}

% Lyapunov exponent limit
\newcommand{\lyaplimit}{\lyap = \lim_{n \to \infty} \frac{1}{n} \sum_{k=0}^{n-1} \log \norm{\mJ(\vx_k)}}

% Possibility manifold definition
\newcommand{\Pdef}{\Pcal = \left\{ (\vz_0, \vc, F) : \orbit(\vz_0, \vc, F) \text{ bounded} \right\}}

% ----------------------------------------------------------------------------
% Psychological Trait Mapping
% ----------------------------------------------------------------------------
\newcommand{\traitO}{O}                         % Openness
\newcommand{\traitV}{V}                         % Volatility
\newcommand{\traitI}{I}                         % Integration
\newcommand{\traitF}{F}                         % Focus

% ----------------------------------------------------------------------------
% Theorem Environments (requires amsthm)
% ----------------------------------------------------------------------------
% \newtheorem{theorem}{Theorem}[section]
% \newtheorem{lemma}[theorem]{Lemma}
% \newtheorem{proposition}[theorem]{Proposition}
% \newtheorem{corollary}[theorem]{Corollary}
% \theoremstyle{definition}
% \newtheorem{definition}[theorem]{Definition}
% \newtheorem{example}[theorem]{Example}
% \theoremstyle{remark}
% \newtheorem{remark}[theorem]{Remark}

% ============================================================================
% End of Macros
% ============================================================================


\title{Base Fractal Dynamics Model}
\author{MindFractal Lab}
\date{\today}

\newtheorem{theorem}{Theorem}[section]
\newtheorem{lemma}[theorem]{Lemma}
\newtheorem{proposition}[theorem]{Proposition}
\newtheorem{corollary}[theorem]{Corollary}
\theoremstyle{definition}
\newtheorem{definition}[theorem]{Definition}
\newtheorem{example}[theorem]{Example}
\theoremstyle{remark}
\newtheorem{remark}[theorem]{Remark}

\begin{document}

\maketitle

\begin{abstract}
This document defines the base 2D and 3D real-valued fractal dynamical systems used in MindFractal Lab. We present the discrete-time nonlinear maps, derive stability conditions, characterize attractor types, and formalize the Lyapunov exponent framework.
\end{abstract}

\tableofcontents

% ----------------------------------------------------------------------------
\section{Introduction}
% ----------------------------------------------------------------------------

The base model is a discrete-time nonlinear dynamical system that exhibits rich behavior including fixed points, limit cycles, chaotic attractors, and fractal basin boundaries. The model is designed to be:

\begin{itemize}
    \item \textbf{Minimal}: Small number of parameters for interpretability
    \item \textbf{Complete}: Captures essential nonlinear phenomena
    \item \textbf{Computationally tractable}: Efficient simulation and analysis
    \item \textbf{Extensible}: Natural generalization to higher dimensions
\end{itemize}

% ----------------------------------------------------------------------------
\section{Model Definition}
% ----------------------------------------------------------------------------

\subsection{2D Dynamical System}

\begin{definition}[2D Fractal Dynamics Map]
The 2D fractal dynamics model is defined by the discrete-time map $f: \R^2 \to \R^2$:
\begin{equation}
    \vx\tnp = f(\vx\tn) = \mA\vx\tn + \mB\tanh(\mW\vx\tn) + \vc
    \label{eq:base_map}
\end{equation}
where:
\begin{itemize}
    \item $\vx\tn \in \R^2$ is the state vector at discrete time $n \in \N$
    \item $\mA \in \R^{2 \times 2}$ is the linear feedback matrix
    \item $\mB \in \R^{2 \times 2}$ is the nonlinear coupling matrix
    \item $\mW \in \R^{2 \times 2}$ is the weight matrix
    \item $\vc \in \R^2$ is the external drive (control parameter)
    \item $\tanh(\cdot)$ is applied element-wise
\end{itemize}
\end{definition}

\subsection{Default Parameters}

The default parameter configuration that produces rich dynamics is:
\begin{equation}
    \mA = \begin{pmatrix} 0.9 & 0 \\ 0 & 0.9 \end{pmatrix}, \quad
    \mB = \begin{pmatrix} 0.2 & 0.3 \\ 0.3 & 0.2 \end{pmatrix}, \quad
    \mW = \begin{pmatrix} 1.0 & 0.1 \\ 0.1 & 1.0 \end{pmatrix}
    \label{eq:default_params}
\end{equation}

\subsection{3D Extension}

\begin{definition}[3D Fractal Dynamics Map]
The 3D extension follows the same structure with $\vx \in \R^3$:
\begin{equation}
    \vx\tnp = \mA\vx\tn + \mB\tanh(\mW\vx\tn) + \vc
    \label{eq:3d_map}
\end{equation}
with default parameters:
\begin{equation}
    \mA = 0.9\mI_3, \quad
    \mB = \begin{pmatrix} 0.2 & 0.1 & 0.1 \\ 0.1 & 0.2 & 0.1 \\ 0.1 & 0.1 & 0.2 \end{pmatrix}, \quad
    \vc = \begin{pmatrix} 0.1 \\ 0.1 \\ 0.1 \end{pmatrix}
\end{equation}
\end{definition}

% ----------------------------------------------------------------------------
\section{Component Interpretation}
% ----------------------------------------------------------------------------

\subsection{Linear Term: $\mA\vx\tn$}

The linear term encodes intrinsic decay or amplification:
\begin{itemize}
    \item \textbf{Diagonal elements} ($A_{ii}$): Self-feedback of each component
    \item \textbf{Off-diagonal elements} ($A_{ij}, i \neq j$): Cross-coupling between components
    \item \textbf{Spectral condition}: Bounded dynamics require $\rho(\mA) < 1$ where $\rho$ denotes spectral radius
\end{itemize}

\begin{proposition}[Boundedness of Linear Part]
If $\rho(\mA) < 1$, then the linear system $\vx\tnp = \mA\vx\tn$ is globally asymptotically stable with $\lim_{n \to \infty} \vx\tn = \vzero$.
\end{proposition}

\subsection{Nonlinear Term: $\mB\tanh(\mW\vx\tn)$}

The nonlinear term provides:
\begin{itemize}
    \item \textbf{Saturation}: Bounded output via $\tanh(\cdot) \in (-1, 1)^d$
    \item \textbf{Neural-like activation}: Sigmoidal response to inputs
    \item \textbf{Multistability}: Enables coexistence of multiple attractors
\end{itemize}

The weight matrix $\mW$ determines input sensitivity, while $\mB$ distributes the nonlinear output.

\subsection{External Drive: $\vc$}

The constant drive vector represents:
\begin{itemize}
    \item Environmental context or external input
    \item Personality traits (via trait-to-parameter mapping)
    \item Primary bifurcation parameter for fractal structure analysis
\end{itemize}

% ----------------------------------------------------------------------------
\section{Fixed Point Analysis}
% ----------------------------------------------------------------------------

\subsection{Fixed Point Equation}

\begin{definition}[Fixed Point]
A fixed point $\xstar$ satisfies:
\begin{equation}
    \xstar = f(\xstar) = \mA\xstar + \mB\tanh(\mW\xstar) + \vc
    \label{eq:fixed_point}
\end{equation}
\end{definition}

Rearranging yields the implicit equation:
\begin{equation}
    (\mI - \mA)\xstar = \mB\tanh(\mW\xstar) + \vc
    \label{eq:fixed_point_implicit}
\end{equation}

\begin{remark}
Equation~\eqref{eq:fixed_point_implicit} is solved numerically via Newton's method with initial guess $\vx_0 = (\mI - \mA)^{-1}\vc$ (the fixed point of the linear part plus constant).
\end{remark}

\subsection{Jacobian Matrix}

\begin{theorem}[Jacobian of the Dynamics Map]
The Jacobian of $f$ at state $\vx$ is:
\begin{equation}
    \mJ(\vx) = \pdv{f}{\vx} = \mA + \mB \cdot \diag\left(\sech^2(\mW\vx)\right) \cdot \mW
    \label{eq:jacobian}
\end{equation}
where $\sech^2(u) = 1 - \tanh^2(u)$ and $\diag(\cdot)$ constructs a diagonal matrix.
\end{theorem}

\begin{proof}
Using the chain rule:
\begin{align}
    \pdv{}{\vx}\left[\mB\tanh(\mW\vx)\right] &= \mB \cdot \pdv{}{\vx}\tanh(\mW\vx) \\
    &= \mB \cdot \diag\left(\pdv{}{\vx}\tanh(\mW\vx)\right) \cdot \mW \\
    &= \mB \cdot \diag\left(\sech^2(\mW\vx)\right) \cdot \mW
\end{align}
Adding the linear term derivative $\pdv{}{\vx}[\mA\vx] = \mA$ gives \eqref{eq:jacobian}.
\end{proof}

\subsection{Linear Stability Criterion}

\begin{theorem}[Local Stability of Fixed Points]
A fixed point $\xstar$ is locally asymptotically stable if and only if all eigenvalues $\mu_i$ of $\mJ(\xstar)$ satisfy:
\begin{equation}
    \abs{\mu_i} < 1 \quad \forall i \in \{1, \ldots, d\}
    \label{eq:stability}
\end{equation}
\end{theorem}

\begin{definition}[Fixed Point Classification]
Based on the eigenvalues $\{\mu_i\}$ of $\mJ(\xstar)$:
\begin{itemize}
    \item \textbf{Stable node}: All $\abs{\mu_i} < 1$, all real
    \item \textbf{Stable spiral}: Complex conjugate pair with $\abs{\mu} < 1$
    \item \textbf{Saddle}: Mixed $\abs{\mu} < 1$ and $\abs{\mu} > 1$
    \item \textbf{Unstable node/spiral}: At least one $\abs{\mu} > 1$
\end{itemize}
\end{definition}

% ----------------------------------------------------------------------------
\section{Lyapunov Exponent Analysis}
% ----------------------------------------------------------------------------

\subsection{Definition}

\begin{definition}[Largest Lyapunov Exponent]
The largest Lyapunov exponent $\lyap$ quantifies the average exponential rate of separation of nearby trajectories:
\begin{equation}
    \lyap = \lim_{n \to \infty} \frac{1}{n} \sum_{k=0}^{n-1} \log \norm{\mJ(\vx_k)}
    \label{eq:lyap_def}
\end{equation}
\end{definition}

More precisely, using tangent vector evolution:
\begin{equation}
    \lyap = \lim_{n \to \infty} \frac{1}{n} \log \norm{\prod_{k=0}^{n-1} \mJ(\vx_k)}
    \label{eq:lyap_product}
\end{equation}

\subsection{Computational Algorithm}

\begin{enumerate}
    \item Initialize tangent vector $\mathbf{v}_0$ with unit norm
    \item For each iteration $k$:
    \begin{enumerate}
        \item Evolve: $\mathbf{v}_{k+1} = \mJ(\vx_k)\mathbf{v}_k$
        \item Record: $s_k = \log\norm{\mathbf{v}_{k+1}}$
        \item Renormalize: $\mathbf{v}_{k+1} \leftarrow \mathbf{v}_{k+1} / \norm{\mathbf{v}_{k+1}}$
    \end{enumerate}
    \item Estimate: $\lyap \approx \frac{1}{n}\sum_{k=0}^{n-1} s_k$
\end{enumerate}

\subsection{Dynamical Classification}

\begin{table}[h]
\centering
\begin{tabular}{|c|c|c|}
\hline
\textbf{Lyapunov Exponent} & \textbf{Dynamics} & \textbf{Behavior} \\
\hline
$\lyap < 0$ & Stable & Convergence to fixed point \\
$\lyap \approx 0$ & Neutral & Periodic or quasiperiodic orbit \\
$\lyap > 0$ & Chaotic & Sensitive dependence on initial conditions \\
\hline
\end{tabular}
\caption{Dynamical classification by Lyapunov exponent}
\label{tab:lyap_class}
\end{table}

\subsection{Full Lyapunov Spectrum}

In $d$ dimensions, the full Lyapunov spectrum $\{\lyap_1, \lyap_2, \ldots, \lyap_d\}$ is computed via QR decomposition of the tangent map product. The spectrum is ordered: $\lyap_1 \geq \lyap_2 \geq \cdots \geq \lyap_d$.

\begin{definition}[Hyperchaos]
A system exhibits \emph{hyperchaos} if at least two Lyapunov exponents are positive:
\begin{equation}
    \lyap_1 > 0 \quad \text{and} \quad \lyap_2 > 0
\end{equation}
\end{definition}

% ----------------------------------------------------------------------------
\section{Attractor Classification}
% ----------------------------------------------------------------------------

\subsection{Fixed Point Attractors}

Stable equilibria satisfying $\lyap < 0$.

\textbf{Condition}: All eigenvalues of $\mJ(\xstar)$ have modulus less than 1.

\subsection{Limit Cycles}

Periodic orbits with period $p$: $f^p(\vx) = \vx$.

\textbf{Condition}: $\lyap \approx 0$ with exactly one zero exponent (along the orbit direction).

\subsection{Quasiperiodic Attractors (Tori)}

Motion on invariant tori with $k$ incommensurate frequencies.

\textbf{Condition}: $k$ zero Lyapunov exponents with remaining exponents negative.

\subsection{Chaotic (Strange) Attractors}

Aperiodic, bounded dynamics with fractal structure.

\textbf{Condition}: $\lyap_1 > 0$ with bounded trajectories.

% ----------------------------------------------------------------------------
\section{Basin of Attraction}
% ----------------------------------------------------------------------------

\begin{definition}[Basin of Attraction]
The basin of attraction of an attractor $\Acal$ is:
\begin{equation}
    \Bcal(\Acal) = \left\{ \vx_0 \in \R^d : \lim_{n \to \infty} f^n(\vx_0) \in \Acal \right\}
    \label{eq:basin}
\end{equation}
\end{definition}

\subsection{Fractal Basin Boundaries}

When multiple attractors coexist, basin boundaries often exhibit fractal structure characterized by:
\begin{enumerate}
    \item \textbf{Self-similarity}: Patterns repeat at multiple scales
    \item \textbf{Non-integer dimension}: Box-counting dimension $D \in (1, 2)$ for 2D systems
    \item \textbf{Sensitivity}: Initial conditions near boundaries yield unpredictable outcomes
\end{enumerate}

\subsection{Box-Counting Dimension}

\begin{definition}[Box-Counting Dimension]
The box-counting dimension of a set $S$ is:
\begin{equation}
    \dimB(S) = \lim_{\epsilon \to 0} \frac{\log N(\epsilon)}{\log(1/\epsilon)}
    \label{eq:box_dim}
\end{equation}
where $N(\epsilon)$ is the number of boxes of side $\epsilon$ needed to cover $S$.
\end{definition}

A basin boundary $\partial\Bcal$ is fractal if $\dimB(\partial\Bcal) > d - 1$.

% ----------------------------------------------------------------------------
\section{Parameter-Space Fractals}
% ----------------------------------------------------------------------------

Varying the control parameter $\vc$ reveals rich bifurcation structure in parameter space.

\begin{definition}[Lyapunov Exponent Map]
The parameter-space Lyapunov map $\Lambda: \R^2 \to \R$ is:
\begin{equation}
    \Lambda(c_1, c_2) = \lyap(\vx_0; c_1, c_2)
    \label{eq:lyap_map}
\end{equation}
where $\lyap$ is computed with fixed initial condition $\vx_0$ and varying $\vc = (c_1, c_2)^T$.
\end{definition}

The resulting visualization reveals:
\begin{itemize}
    \item \textbf{Stability regions}: Parameter sets leading to $\lyap < 0$
    \item \textbf{Chaotic regions}: Parameter sets yielding $\lyap > 0$
    \item \textbf{Fractal boundaries}: Self-similar bifurcation frontiers
\end{itemize}

% ----------------------------------------------------------------------------
\section{Metastability}
% ----------------------------------------------------------------------------

\begin{definition}[Metastable Regime]
A trajectory is in a metastable regime if it dwells near a saddle point or basin boundary for extended time before settling to an attractor.
\end{definition}

The characteristic dwell time scales as:
\begin{equation}
    \tau \sim d(\vx_0, \partial\Bcal)^{-\alpha}
    \label{eq:dwell}
\end{equation}
where $d(\vx_0, \partial\Bcal)$ is the distance to the basin boundary and $\alpha > 0$ depends on local geometry.

% ----------------------------------------------------------------------------
\section{Energy Function}
% ----------------------------------------------------------------------------

An energy-like function monitoring system state:
\begin{equation}
    E(\vx) = \norm{\vx}^2 + \gamma \norm{\mW\vx}^2
    \label{eq:energy}
\end{equation}
where $\gamma \geq 0$ is a weighting factor.

\begin{remark}
This is not a Lyapunov function in the strict sense but provides useful diagnostics for trajectory boundedness.
\end{remark}

\end{document}
