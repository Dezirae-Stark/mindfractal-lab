% ============================================================================
% Calabi-Yau Extension — Complex High-Dimensional Dynamics
% MindFractal Lab Mathematical Documentation
% ============================================================================

\documentclass[11pt, a4paper]{article}

\usepackage{amsmath, amssymb, amsthm}
\usepackage{mathtools}
\usepackage{physics}
\usepackage{hyperref}

% ============================================================================
% MindFractal Lab — LaTeX Macros
% Common mathematical notation for the Fractal Consciousness Model
% ============================================================================

% ----------------------------------------------------------------------------
% Number Sets and Spaces
% ----------------------------------------------------------------------------
\newcommand{\R}{\mathbb{R}}                     % Real numbers
\newcommand{\C}{\mathbb{C}}                     % Complex numbers
\newcommand{\Z}{\mathbb{Z}}                     % Integers
\newcommand{\N}{\mathbb{N}}                     % Natural numbers

% State/parameter spaces
\newcommand{\Pcal}{\mathcal{P}}                 % Possibility manifold
\newcommand{\Bcal}{\mathcal{B}}                 % Basin of attraction
\newcommand{\Acal}{\mathcal{A}}                 % Attractor set
\newcommand{\Tcal}{\mathcal{T}}                 % Timeline space
\newcommand{\Fcal}{\mathcal{F}}                 % Function family

% ----------------------------------------------------------------------------
% Vectors and Matrices
% ----------------------------------------------------------------------------
\newcommand{\vx}{\mathbf{x}}                    % State vector (real)
\newcommand{\vz}{\mathbf{z}}                    % State vector (complex)
\newcommand{\vc}{\mathbf{c}}                    % Parameter vector
\newcommand{\veta}{\boldsymbol{\eta}}           % Noise vector
\newcommand{\vzero}{\mathbf{0}}                 % Zero vector

\newcommand{\mA}{\mathbf{A}}                    % Linear feedback matrix
\newcommand{\mB}{\mathbf{B}}                    % Nonlinear coupling matrix
\newcommand{\mW}{\mathbf{W}}                    % Weight matrix
\newcommand{\mU}{\mathbf{U}}                    % Unitary matrix
\newcommand{\mH}{\mathbf{H}}                    % Hermitian matrix
\newcommand{\mI}{\mathbf{I}}                    % Identity matrix
\newcommand{\mJ}{\mathbf{J}}                    % Jacobian matrix

% ----------------------------------------------------------------------------
% Operators and Functions
% ----------------------------------------------------------------------------
\newcommand{\diag}{\operatorname{diag}}         % Diagonal operator
\newcommand{\sech}{\operatorname{sech}}         % Hyperbolic secant
\newcommand{\orbit}{\operatorname{orbit}}       % Orbit function
\newcommand{\sgn}{\operatorname{sgn}}           % Sign function
\newcommand{\tr}{\operatorname{tr}}             % Trace
\newcommand{\rank}{\operatorname{rank}}         % Rank
\newcommand{\spec}{\operatorname{spec}}         % Spectrum

% Norms
\newcommand{\norm}[1]{\left\| #1 \right\|}      % Generic norm
\newcommand{\abs}[1]{\left| #1 \right|}         % Absolute value

% Element-wise product (Hadamard)
\newcommand{\had}{\odot}                        % Hadamard product

% ----------------------------------------------------------------------------
% Dynamical Systems Notation
% ----------------------------------------------------------------------------
\newcommand{\lyap}{\lambda}                     % Lyapunov exponent
\newcommand{\lyapmax}{\lambda_{\max}}           % Maximum Lyapunov exponent
\newcommand{\lyapspec}{\boldsymbol{\lambda}}    % Lyapunov spectrum

% Fixed point notation
\newcommand{\xstar}{\vx^*}                      % Fixed point (real)
\newcommand{\zstar}{\vz^*}                      % Fixed point (complex)

% Time indices
\newcommand{\tn}{_{n}}                          % Time n subscript
\newcommand{\tnp}{_{n+1}}                       % Time n+1 subscript

% ----------------------------------------------------------------------------
% Calabi-Yau Extension Notation
% ----------------------------------------------------------------------------
\newcommand{\CY}{\text{CY}}                     % Calabi-Yau abbreviation
\newcommand{\eps}{\varepsilon}                  % Epsilon for nonlinearity

% Update rules
\newcommand{\Ftanh}{F_{\tanh}}                  % Tanh update rule
\newcommand{\Fsigmoid}{F_{\sigma}}              % Sigmoid update rule
\newcommand{\Fthree}{F_{3D}}                    % 3D update rule
\newcommand{\Fcy}{F_{\CY}}                      % CY update rule

% ----------------------------------------------------------------------------
% Metric and Distance Notation
% ----------------------------------------------------------------------------
\newcommand{\dP}{d_{\Pcal}}                     % Distance on possibility manifold
\newcommand{\dH}{d_H}                           % Hausdorff distance
\newcommand{\dimH}{\dim_H}                      % Hausdorff dimension
\newcommand{\dimB}{\dim_B}                      % Box-counting dimension

% ----------------------------------------------------------------------------
% Common Expressions
% ----------------------------------------------------------------------------
% State update equation
\newcommand{\stateupdate}{\vx\tnp = \mA\vx\tn + \mB\tanh(\mW\vx\tn) + \vc}

% Complex state update
\newcommand{\cstateupdate}{\vz\tnp = \mU\vz\tn + \eps(\vz\tn \had \vz\tn) + \vc}

% Jacobian expression
\newcommand{\jacobian}{\mJ(\vx) = \mA + \mB \cdot \diag\left(\sech^2(\mW\vx)\right) \cdot \mW}

% Lyapunov exponent limit
\newcommand{\lyaplimit}{\lyap = \lim_{n \to \infty} \frac{1}{n} \sum_{k=0}^{n-1} \log \norm{\mJ(\vx_k)}}

% Possibility manifold definition
\newcommand{\Pdef}{\Pcal = \left\{ (\vz_0, \vc, F) : \orbit(\vz_0, \vc, F) \text{ bounded} \right\}}

% ----------------------------------------------------------------------------
% Psychological Trait Mapping
% ----------------------------------------------------------------------------
\newcommand{\traitO}{O}                         % Openness
\newcommand{\traitV}{V}                         % Volatility
\newcommand{\traitI}{I}                         % Integration
\newcommand{\traitF}{F}                         % Focus

% ----------------------------------------------------------------------------
% Theorem Environments (requires amsthm)
% ----------------------------------------------------------------------------
% \newtheorem{theorem}{Theorem}[section]
% \newtheorem{lemma}[theorem]{Lemma}
% \newtheorem{proposition}[theorem]{Proposition}
% \newtheorem{corollary}[theorem]{Corollary}
% \theoremstyle{definition}
% \newtheorem{definition}[theorem]{Definition}
% \newtheorem{example}[theorem]{Example}
% \theoremstyle{remark}
% \newtheorem{remark}[theorem]{Remark}

% ============================================================================
% End of Macros
% ============================================================================


\title{Calabi-Yau Inspired Complex Dynamics}
\author{MindFractal Lab}
\date{\today}

\newtheorem{theorem}{Theorem}[section]
\newtheorem{lemma}[theorem]{Lemma}
\newtheorem{proposition}[theorem]{Proposition}
\newtheorem{corollary}[theorem]{Corollary}
\theoremstyle{definition}
\newtheorem{definition}[theorem]{Definition}
\newtheorem{example}[theorem]{Example}
\theoremstyle{remark}
\newtheorem{remark}[theorem]{Remark}

\begin{document}

\maketitle

\begin{abstract}
We extend the base fractal dynamics model to complex-valued high-dimensional spaces inspired by Calabi-Yau manifold geometry. The CY-extension introduces unitary evolution operators, Hermitian structure, and complex nonlinear dynamics that preserve certain geometric properties. This provides a natural framework for modeling higher-dimensional possibility spaces.
\end{abstract}

\tableofcontents

% ----------------------------------------------------------------------------
\section{Introduction}
% ----------------------------------------------------------------------------

Calabi-Yau manifolds are compact K\"ahler manifolds with vanishing first Chern class, playing a central role in string theory as compactification spaces. While our model is not a literal Calabi-Yau space, we draw inspiration from key properties:

\begin{itemize}
    \item \textbf{Complex structure}: State space is $\C^k$ rather than $\R^d$
    \item \textbf{Hermitian metrics}: Inner products compatible with complex structure
    \item \textbf{Unitary transformations}: Evolution that preserves norms
    \item \textbf{Rich topology}: High-dimensional manifolds with interesting geometry
\end{itemize}

\begin{remark}[Disclaimer]
This extension is a \emph{conceptual model} using CY-inspired mathematics, not a literal physics model of Calabi-Yau compactification. The connection is metaphorical and computational rather than physical.
\end{remark}

% ----------------------------------------------------------------------------
\section{Complex Dynamical System}
% ----------------------------------------------------------------------------

\subsection{CY-Inspired Update Rule}

\begin{definition}[CY Complex Dynamics]
The Calabi-Yau inspired dynamical system is defined by:
\begin{equation}
    \vz\tnp = \mU \vz\tn + \eps \left(\vz\tn \had \vz\tn\right) + \vc
    \label{eq:cy_update}
\end{equation}
where:
\begin{itemize}
    \item $\vz\tn \in \C^k$ is the complex state vector at time $n$
    \item $\mU \in \C^{k \times k}$ is a unitary matrix ($\mU^\dagger \mU = \mI$)
    \item $\eps \in \C$ is the nonlinearity strength parameter
    \item $\had$ denotes the Hadamard (element-wise) product
    \item $\vc \in \C^k$ is the complex parameter vector
\end{itemize}
\end{definition}

\subsection{Component Interpretation}

\subsubsection{Unitary Evolution: $\mU\vz\tn$}

The unitary matrix $\mU$ provides:
\begin{itemize}
    \item \textbf{Norm preservation}: $\norm{\mU\vz} = \norm{\vz}$ for all $\vz$
    \item \textbf{Rotation in complex space}: Eigenvalues lie on the unit circle
    \item \textbf{Quantum-inspired dynamics}: Analogous to unitary evolution in quantum mechanics
\end{itemize}

\begin{proposition}[Spectral Properties of $\mU$]
All eigenvalues $\mu_i$ of a unitary matrix $\mU$ satisfy $\abs{\mu_i} = 1$. Hence $\mu_i = e^{i\theta_i}$ for real phases $\theta_i$.
\end{proposition}

\subsubsection{Nonlinear Term: $\eps(\vz\tn \had \vz\tn)$}

The Hadamard square provides:
\begin{itemize}
    \item \textbf{Complex nonlinearity}: $(z_1^2, z_2^2, \ldots, z_k^2)^T$
    \item \textbf{Mode coupling}: Quadratic self-interaction of each component
    \item \textbf{Mandelbrot-like dynamics}: Reduces to $z \mapsto z^2 + c$ for $k=1$, $\mU=\mI$
\end{itemize}

\subsubsection{Parameter Vector: $\vc$}

The complex parameter vector controls:
\begin{itemize}
    \item System offset and bias
    \item Bifurcation structure in complex parameter space
    \item Connection to Julia/Mandelbrot set theory
\end{itemize}

% ----------------------------------------------------------------------------
\section{Alternative CY Update Rules}
% ----------------------------------------------------------------------------

\subsection{Hermitian-Weighted Update}

\begin{definition}[Hermitian CY Dynamics]
An alternative formulation using Hermitian structure:
\begin{equation}
    \vz\tnp = \mH \vz\tn + \mB \tanh(\mU \vz\tn) + \vc
    \label{eq:cy_hermitian}
\end{equation}
where:
\begin{itemize}
    \item $\mH \in \C^{k \times k}$ is Hermitian ($\mH^\dagger = \mH$)
    \item $\mU \in \C^{k \times k}$ is unitary
    \item $\mB \in \C^{k \times k}$ is the coupling matrix
\end{itemize}
\end{definition}

\begin{remark}
The $\tanh$ applied to complex arguments is interpreted component-wise on real and imaginary parts, or via the complex extension $\tanh(z) = \frac{e^z - e^{-z}}{e^z + e^{-z}}$.
\end{remark}

\subsection{Coupled Real-Imaginary Dynamics}

\begin{definition}[Split Complex Dynamics]
Writing $\vz = \vx + i\vy$ with $\vx, \vy \in \R^k$:
\begin{align}
    \vx\tnp &= \mA\vx\tn - \mC\vy\tn + \mB\tanh(\mW\vx\tn) + \Re(\vc) \\
    \vy\tnp &= \mC\vx\tn + \mA\vy\tn + \mB\tanh(\mW\vy\tn) + \Im(\vc)
\end{align}
where $\mA, \mC, \mB, \mW \in \R^{k \times k}$.
\end{definition}

This formulation makes the coupling between real and imaginary parts explicit.

% ----------------------------------------------------------------------------
\section{Geometric Structure}
% ----------------------------------------------------------------------------

\subsection{Hermitian Inner Product}

\begin{definition}[Hermitian Inner Product]
The natural inner product on $\C^k$ is:
\begin{equation}
    \langle \vz, \vw \rangle = \vz^\dagger \vw = \sum_{j=1}^k \overline{z_j} w_j
    \label{eq:hermitian_ip}
\end{equation}
with induced norm $\norm{\vz} = \sqrt{\langle \vz, \vz \rangle}$.
\end{definition}

\subsection{K\"ahler-like Structure}

While not a true K\"ahler manifold, we can define analogous structures:

\begin{definition}[Symplectic Form]
Define $\omega: \C^k \times \C^k \to \R$ by:
\begin{equation}
    \omega(\vz, \vw) = \Im\langle \vz, \vw \rangle = \sum_{j=1}^k (x_j v_j - y_j u_j)
    \label{eq:symplectic}
\end{equation}
where $\vz = \vx + i\vy$ and $\vw = \vu + i\vv$.
\end{definition}

\begin{proposition}[Symplectic Preservation]
The unitary part of the dynamics preserves $\omega$:
\begin{equation}
    \omega(\mU\vz, \mU\vw) = \omega(\vz, \vw)
\end{equation}
\end{proposition}

% ----------------------------------------------------------------------------
\section{Jacobian and Stability}
% ----------------------------------------------------------------------------

\subsection{Complex Jacobian}

\begin{theorem}[Jacobian of CY Dynamics]
For the update rule \eqref{eq:cy_update}, the Jacobian at $\vz$ is:
\begin{equation}
    \mJ(\vz) = \mU + 2\eps \diag(\vz)
    \label{eq:cy_jacobian}
\end{equation}
where $\diag(\vz)$ is the diagonal matrix with entries $z_1, \ldots, z_k$.
\end{theorem}

\begin{proof}
The derivative of the unitary term is $\mU$. For the nonlinear term:
\begin{equation}
    \pdv{}{z_j}(z_i^2) = 2z_i \delta_{ij}
\end{equation}
Hence $\pdv{}{\vz}(\vz \had \vz) = 2\diag(\vz)$.
\end{proof}

\subsection{Stability Analysis}

\begin{theorem}[Fixed Point Stability]
A fixed point $\zstar$ is stable if all eigenvalues $\mu_i$ of $\mJ(\zstar)$ satisfy $\abs{\mu_i} < 1$.
\end{theorem}

For the CY system with small $\eps$:
\begin{equation}
    \mJ(\zstar) \approx \mU + O(\eps)
\end{equation}
Since eigenvalues of $\mU$ have $\abs{\mu} = 1$, small perturbations can push eigenvalues inside or outside the unit circle, leading to bifurcations.

% ----------------------------------------------------------------------------
\section{Complex Lyapunov Exponents}
% ----------------------------------------------------------------------------

\subsection{Definition}

\begin{definition}[Complex Lyapunov Exponent]
The largest Lyapunov exponent for complex dynamics:
\begin{equation}
    \lyap = \lim_{n \to \infty} \frac{1}{n} \sum_{k=0}^{n-1} \log \norm{\mJ(\vz_k)}
    \label{eq:cy_lyap}
\end{equation}
where the norm is the operator norm induced by the Hermitian inner product.
\end{definition}

\subsection{Lyapunov Spectrum}

For $k$-dimensional complex dynamics (viewed as $2k$-dimensional real), the full spectrum has $2k$ exponents. However, due to complex structure, they often appear in conjugate pairs.

% ----------------------------------------------------------------------------
\section{Connection to Mandelbrot Dynamics}
% ----------------------------------------------------------------------------

\subsection{One-Dimensional Reduction}

Setting $k = 1$, $\mU = 1$, the CY dynamics reduce to:
\begin{equation}
    z\tnp = z\tn + \eps z_n^2 + c = z_n^2 + c \quad (\text{for } \eps = 1)
    \label{eq:mandelbrot}
\end{equation}
This is the famous Mandelbrot iteration.

\begin{definition}[Mandelbrot Set]
The Mandelbrot set is:
\begin{equation}
    \mathcal{M} = \left\{ c \in \C : \sup_{n} \abs{z_n} < \infty \text{ with } z_0 = 0 \right\}
    \label{eq:mandelbrot_set}
\end{equation}
\end{definition}

\begin{definition}[Julia Set]
The filled Julia set for parameter $c$ is:
\begin{equation}
    K_c = \left\{ z_0 \in \C : \sup_{n} \abs{z_n} < \infty \right\}
    \label{eq:julia_set}
\end{equation}
The Julia set $J_c = \partial K_c$ is its boundary.
\end{definition}

\subsection{Higher-Dimensional Generalization}

The CY extension can be viewed as a vector-valued generalization of Mandelbrot dynamics, with:
\begin{itemize}
    \item Component-wise quadratic nonlinearity
    \item Linear mixing via unitary transformation
    \item Rich fractal structure in $\C^k$
\end{itemize}

% ----------------------------------------------------------------------------
\section{Projections and Slices}
% ----------------------------------------------------------------------------

\subsection{2D Projections}

For visualization, project $\C^k$ dynamics to $\C$ or $\R^2$:

\begin{definition}[Complex Projection]
A projection $\pi: \C^k \to \C$ via inner product:
\begin{equation}
    \pi(\vz) = \langle \mathbf{v}, \vz \rangle = \mathbf{v}^\dagger \vz
\end{equation}
for fixed projection vector $\mathbf{v} \in \C^k$ with $\norm{\mathbf{v}} = 1$.
\end{definition}

\begin{definition}[Real 2D Projection]
A projection $P: \C^k \to \R^2$ selecting components:
\begin{equation}
    P(\vz) = (\Re(z_i), \Im(z_j))
\end{equation}
for chosen indices $i, j$.
\end{definition}

\subsection{Parameter Slices}

To visualize the $(2k)$-dimensional parameter space:

\begin{definition}[Parameter Slice]
Fix $k-1$ complex parameters and vary one:
\begin{equation}
    \vc(t) = \vc_0 + t \cdot \mathbf{e}_j, \quad t \in \C
\end{equation}
where $\mathbf{e}_j$ is the $j$-th standard basis vector.
\end{definition}

This produces 2D slices through parameter space showing fractal structure analogous to the Mandelbrot set.

% ----------------------------------------------------------------------------
\section{Numerical Considerations}
% ----------------------------------------------------------------------------

\subsection{Boundedness Criterion}

\begin{definition}[Escape Radius]
For typical quadratic maps, define escape radius $R$:
\begin{equation}
    R = \max\left(2, \norm{\vc}\right)
\end{equation}
If $\norm{\vz_n} > R$, the orbit escapes to infinity.
\end{definition}

\subsection{Iteration Parameters}

Typical computational parameters:
\begin{itemize}
    \item Maximum iterations: $N_{\max} = 1000$
    \item Escape threshold: $R = 10$ or $R = 100$
    \item Convergence threshold: $\norm{\vz_n - \vz_{n-p}} < \delta$ for period $p$
\end{itemize}

\subsection{Coloring Schemes}

For visualization:
\begin{itemize}
    \item \textbf{Escape time}: Color by number of iterations to escape
    \item \textbf{Lyapunov}: Color by estimated Lyapunov exponent
    \item \textbf{Orbit trap}: Color by distance to geometric shapes
    \item \textbf{Period}: Color by detected orbit period
\end{itemize}

% ----------------------------------------------------------------------------
\section{Physical and Metaphorical Interpretations}
% ----------------------------------------------------------------------------

\subsection{String Theory Analogy}

In string theory, Calabi-Yau manifolds are 6-dimensional (3 complex dimensions) compactification spaces. Our model with $k = 3$ can be viewed as dynamics on a \emph{toy model} of such a space.

\begin{itemize}
    \item $\C^3$: Analog of 6 compactified dimensions
    \item Unitary evolution: Preserves ``quantum'' structure
    \item Nonlinearity: Self-interaction of modes
\end{itemize}

\subsection{Consciousness Modeling}

The complex structure provides:
\begin{itemize}
    \item \textbf{Amplitude and phase}: States have magnitude (intensity) and phase (quality)
    \item \textbf{Interference}: Complex superposition enables constructive/destructive dynamics
    \item \textbf{Holomorphic structure}: Richer function theory than real case
\end{itemize}

\begin{remark}
These interpretations are metaphorical. The model is a mathematical tool, not a literal theory of physics or consciousness.
\end{remark}

\end{document}
