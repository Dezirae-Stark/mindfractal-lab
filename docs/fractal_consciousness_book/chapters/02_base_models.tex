% ============================================================================
% Chapter 2: Base Models
% ============================================================================

\chapter{Base Dynamical Models}
\label{ch:base_models}

This chapter presents the foundational 2D and 3D real-valued dynamical systems that form the core of the MindFractal framework.

% ----------------------------------------------------------------------------
\section{The 2D Fractal Dynamics Model}
% ----------------------------------------------------------------------------

\subsection{Model Definition}

\begin{definition}[2D Fractal Dynamics]
The 2D fractal dynamics model is the discrete-time map $f: \R^2 \to \R^2$:
\begin{equation}
    \vx_{n+1} = f(\vx_n) = \mA\vx_n + \mB\tanh(\mW\vx_n) + \vc
    \label{eq:2d_map}
\end{equation}
where:
\begin{align}
    \vx_n &= \begin{pmatrix} x_1 \\ x_2 \end{pmatrix} \in \R^2 && \text{(state vector)} \\
    \mA &\in \R^{2 \times 2} && \text{(linear feedback matrix)} \\
    \mB &\in \R^{2 \times 2} && \text{(nonlinear coupling matrix)} \\
    \mW &\in \R^{2 \times 2} && \text{(weight matrix)} \\
    \vc &\in \R^2 && \text{(external drive)}
\end{align}
and $\tanh$ is applied element-wise.
\end{definition}

\subsection{Default Parameters}

The default configuration producing rich dynamics:
\begin{equation}
    \mA = \begin{pmatrix} 0.9 & 0 \\ 0 & 0.9 \end{pmatrix}, \quad
    \mB = \begin{pmatrix} 0.2 & 0.3 \\ 0.3 & 0.2 \end{pmatrix}, \quad
    \mW = \begin{pmatrix} 1.0 & 0.1 \\ 0.1 & 1.0 \end{pmatrix}
\end{equation}

These parameters ensure:
\begin{itemize}
    \item $\rho(\mA) = 0.9 < 1$: Bounded linear dynamics
    \item Off-diagonal coupling in $\mB$: Component interaction
    \item Near-identity $\mW$: Local nonlinear response
\end{itemize}

% ----------------------------------------------------------------------------
\section{Component Analysis}
% ----------------------------------------------------------------------------

\subsection{Linear Term: $\mA\vx$}

The linear term provides:
\begin{itemize}
    \item \textbf{Contraction}: $\rho(\mA) < 1$ ensures bounded trajectories
    \item \textbf{Rotation}: Off-diagonal elements induce spiraling
    \item \textbf{Timescale}: $\log(1/\rho(\mA))$ sets relaxation rate
\end{itemize}

\begin{proposition}
If $\rho(\mA) < 1$, then for the linear system $\vx_{n+1} = \mA\vx_n$:
\begin{equation}
    \lim_{n \to \infty} \vx_n = \vzero
\end{equation}
with exponential convergence rate $\rho(\mA)$.
\end{proposition}

\subsection{Nonlinear Term: $\mB\tanh(\mW\vx)$}

The hyperbolic tangent provides:
\begin{itemize}
    \item \textbf{Saturation}: $\tanh(u) \to \pm 1$ as $u \to \pm\infty$
    \item \textbf{Sigmoidal response}: Smooth transition between extremes
    \item \textbf{Neural analogy}: Similar to activation functions in neural networks
\end{itemize}

Key properties of $\tanh$:
\begin{align}
    \tanh(0) &= 0 \\
    \frac{d}{du}\tanh(u) &= \sech^2(u) = 1 - \tanh^2(u) \\
    \sech^2(0) &= 1, \quad \sech^2(u) \to 0 \text{ as } |u| \to \infty
\end{align}

\subsection{External Drive: $\vc$}

The constant drive vector:
\begin{itemize}
    \item Shifts the fixed point location
    \item Acts as bifurcation parameter
    \item Represents environmental context or personality traits
\end{itemize}

% ----------------------------------------------------------------------------
\section{Fixed Point Analysis}
% ----------------------------------------------------------------------------

\subsection{Fixed Point Equation}

A fixed point $\xstar$ satisfies:
\begin{equation}
    \xstar = \mA\xstar + \mB\tanh(\mW\xstar) + \vc
\end{equation}

Rearranging:
\begin{equation}
    (\mI - \mA)\xstar = \mB\tanh(\mW\xstar) + \vc
    \label{eq:fp_implicit}
\end{equation}

Since $\mI - \mA$ is invertible when $1 \notin \sigma(\mA)$:
\begin{equation}
    \xstar = (\mI - \mA)^{-1}\left[\mB\tanh(\mW\xstar) + \vc\right]
\end{equation}

This implicit equation is solved numerically via Newton's method.

\subsection{Newton's Method for Fixed Points}

\begin{algorithm}[H]
\caption{Newton's Method for Fixed Points}
\begin{algorithmic}[1]
\STATE Initialize $\vx^{(0)} = (\mI - \mA)^{-1}\vc$
\FOR{$k = 0, 1, 2, \ldots$ until convergence}
    \STATE $g(\vx) = \vx - \mA\vx - \mB\tanh(\mW\vx) - \vc$
    \STATE $\mJ_g = \mI - \mA - \mB \cdot \diag(\sech^2(\mW\vx^{(k)})) \cdot \mW$
    \STATE $\vx^{(k+1)} = \vx^{(k)} - \mJ_g^{-1} g(\vx^{(k)})$
\ENDFOR
\RETURN $\xstar = \vx^{(k)}$
\end{algorithmic}
\end{algorithm}

\subsection{Jacobian Matrix}

\begin{theorem}[Jacobian of the Dynamics]
The Jacobian of $f$ at $\vx$ is:
\begin{equation}
    \mJ(\vx) = \mA + \mB \cdot \diag\left(\sech^2(\mW\vx)\right) \cdot \mW
    \label{eq:jacobian}
\end{equation}
\end{theorem}

\begin{proof}
The derivative of the linear term is $\mA$. For the nonlinear term, using the chain rule:
\begin{align}
    \frac{\partial}{\partial \vx}\left[\mB\tanh(\mW\vx)\right]
    &= \mB \cdot \frac{\partial}{\partial \vx}\tanh(\mW\vx) \\
    &= \mB \cdot \diag\left(\frac{d\tanh}{du}\bigg|_{u=\mW\vx}\right) \cdot \mW \\
    &= \mB \cdot \diag\left(\sech^2(\mW\vx)\right) \cdot \mW
\end{align}
\end{proof}

\subsection{Stability Criterion}

\begin{theorem}[Linear Stability]
A fixed point $\xstar$ is locally asymptotically stable if and only if all eigenvalues $\mu_i$ of $\mJ(\xstar)$ satisfy:
\begin{equation}
    |\mu_i| < 1 \quad \forall i
\end{equation}
\end{theorem}

\begin{definition}[Fixed Point Classification]
Based on eigenvalues of $\mJ(\xstar)$:
\begin{itemize}
    \item \textbf{Stable node}: All $|\mu_i| < 1$, all real
    \item \textbf{Stable spiral}: Complex conjugate pair with $|\mu| < 1$
    \item \textbf{Saddle}: Mixed $|\mu| < 1$ and $|\mu| > 1$
    \item \textbf{Unstable}: At least one $|\mu| > 1$
\end{itemize}
\end{definition}

% ----------------------------------------------------------------------------
\section{Lyapunov Exponents}
% ----------------------------------------------------------------------------

\subsection{Definition}

\begin{definition}[Largest Lyapunov Exponent]
The largest Lyapunov exponent quantifies average exponential divergence of nearby trajectories:
\begin{equation}
    \lyap = \lim_{n \to \infty} \frac{1}{n} \sum_{k=0}^{n-1} \log \norm{\mJ(\vx_k)}
    \label{eq:lyap}
\end{equation}
\end{definition}

\subsection{Interpretation}

\begin{table}[h]
\centering
\begin{tabular}{cll}
\toprule
$\lyap$ & Dynamics & Psychological Interpretation \\
\midrule
$< 0$ & Stable & Focused, convergent thinking \\
$\approx 0$ & Neutral/Periodic & Rhythmic, cyclic patterns \\
$> 0$ & Chaotic & Creative, divergent thinking \\
\bottomrule
\end{tabular}
\caption{Lyapunov exponent interpretation}
\end{table}

\subsection{Computational Algorithm}

\begin{algorithm}[H]
\caption{Lyapunov Exponent Computation}
\begin{algorithmic}[1]
\STATE Initialize $\vx \leftarrow \vx_0$, $\mathbf{v} \leftarrow$ random unit vector, $S \leftarrow 0$
\STATE Transient: iterate $N_{\text{trans}}$ times without accumulating
\FOR{$k = 1$ to $N$}
    \STATE $\mJ \leftarrow \mA + \mB \cdot \diag(\sech^2(\mW\vx)) \cdot \mW$
    \STATE $\mathbf{v} \leftarrow \mJ \cdot \mathbf{v}$
    \STATE $S \leftarrow S + \log\norm{\mathbf{v}}$
    \STATE $\mathbf{v} \leftarrow \mathbf{v} / \norm{\mathbf{v}}$
    \STATE $\vx \leftarrow f(\vx)$
\ENDFOR
\RETURN $\lyap = S / N$
\end{algorithmic}
\end{algorithm}

% ----------------------------------------------------------------------------
\section{Attractor Types}
% ----------------------------------------------------------------------------

\subsection{Fixed Point Attractors}

When $\lyap < 0$, trajectories converge to a stable fixed point.

\textbf{Psychological interpretation}: Persistent mood states, stable attention focus, meditative equilibrium.

\subsection{Limit Cycles}

Periodic orbits with $\vx_{n+p} = \vx_n$ for period $p$.

\textbf{Psychological interpretation}: Rumination, mood cycling, circadian patterns.

\subsection{Strange Attractors}

Bounded, aperiodic dynamics with $\lyap > 0$ and fractal structure.

\textbf{Psychological interpretation}: Creative flow, rapid association, flexible cognition.

% ----------------------------------------------------------------------------
\section{Basin of Attraction}
% ----------------------------------------------------------------------------

\begin{definition}[Basin of Attraction]
The basin of attraction of attractor $\Acal$:
\begin{equation}
    \Bcal(\Acal) = \left\{ \vx_0 : \lim_{n \to \infty} f^n(\vx_0) \in \Acal \right\}
\end{equation}
\end{definition}

\subsection{Fractal Boundaries}

When multiple attractors coexist, basin boundaries often exhibit:
\begin{itemize}
    \item Self-similar structure at all scales
    \item Non-integer box-counting dimension
    \item Sensitive dependence on initial conditions
\end{itemize}

\textbf{Psychological interpretation}: Near basin boundaries, small perturbations can trigger large state changes---analogous to mood instability or decision thresholds.

% ----------------------------------------------------------------------------
\section{The 3D Extension}
% ----------------------------------------------------------------------------

\subsection{3D Model Definition}

\begin{definition}[3D Fractal Dynamics]
The 3D extension:
\begin{equation}
    \vx_{n+1} = \mA\vx_n + \mB\tanh(\mW\vx_n) + \vc, \quad \vx \in \R^3
\end{equation}
with default parameters:
\begin{equation}
    \mA = 0.9\mI_3, \quad
    \mB = \begin{pmatrix} 0.2 & 0.1 & 0.1 \\ 0.1 & 0.2 & 0.1 \\ 0.1 & 0.1 & 0.2 \end{pmatrix}
\end{equation}
\end{definition}

\subsection{Lyapunov Spectrum}

In 3D, we compute three Lyapunov exponents $\lyap_1 \geq \lyap_2 \geq \lyap_3$:
\begin{itemize}
    \item $\lyap_1 > 0$, $\lyap_2 < 0$, $\lyap_3 < 0$: Chaos
    \item $\lyap_1 > 0$, $\lyap_2 > 0$: Hyperchaos
    \item $\lyap_1 \approx 0$: Quasiperiodic torus
\end{itemize}

\subsection{3D Visualization}

3D dynamics enable:
\begin{itemize}
    \item Richer attractor geometry (strange attractors with 3D structure)
    \item More complex bifurcation sequences
    \item Higher-dimensional parameter exploration
\end{itemize}

% ----------------------------------------------------------------------------
\section{Energy Function}
% ----------------------------------------------------------------------------

\begin{definition}[Energy]
An energy-like function for monitoring state:
\begin{equation}
    E(\vx) = \norm{\vx}^2
\end{equation}
\end{definition}

This is not a Lyapunov function in the strict sense but provides useful diagnostics:
\begin{itemize}
    \item $E$ bounded $\Rightarrow$ trajectory bounded
    \item $E$ increasing $\Rightarrow$ potential escape
    \item $E$ oscillating $\Rightarrow$ periodic or quasiperiodic dynamics
\end{itemize}

% ----------------------------------------------------------------------------
\section{Summary}
% ----------------------------------------------------------------------------

The base models provide:
\begin{enumerate}
    \item A minimal nonlinear map with rich dynamics
    \item Rigorous fixed point and stability analysis
    \item Lyapunov characterization of dynamical regimes
    \item Fractal basin boundaries enabling metastability
    \item Natural extension to 3D with richer spectrum
\end{enumerate}

These real-valued models form the foundation for the complex extensions in Chapter 3.
