% ============================================================================
% Chapter 3: Calabi-Yau Inspired Dynamics
% ============================================================================

\chapter{Calabi-Yau Inspired Complex Dynamics}
\label{ch:cy_dynamics}

This chapter extends the base models to complex-valued high-dimensional spaces, drawing inspiration from Calabi-Yau manifold geometry.

% ----------------------------------------------------------------------------
\section{Motivation for Complex Extension}
% ----------------------------------------------------------------------------

The real-valued base models capture essential nonlinear dynamics, but complex spaces offer:

\begin{itemize}
    \item \textbf{Richer structure}: Complex multiplication, holomorphic functions
    \item \textbf{Natural rotations}: Unitary transformations preserve norms
    \item \textbf{Mandelbrot connection}: Link to classical complex dynamics
    \item \textbf{Higher dimensions}: Model more degrees of freedom
\end{itemize}

\subsection{Calabi-Yau Inspiration}

Calabi-Yau manifolds are compact K\"ahler manifolds with vanishing first Chern class. In string theory, they provide compactification spaces for extra dimensions. While our model is not a literal CY space, we adopt key features:

\begin{itemize}
    \item Complex coordinates on state space
    \item Unitary (norm-preserving) evolution
    \item Hermitian metric structure
\end{itemize}

\begin{remark}[Disclaimer]
This is a \emph{conceptual model} inspired by CY geometry, not a physical theory. The connection is metaphorical and computational.
\end{remark}

% ----------------------------------------------------------------------------
\section{The CY Complex Dynamics Model}
% ----------------------------------------------------------------------------

\subsection{Primary Definition}

\begin{definition}[CY Complex Dynamics]
The Calabi-Yau inspired update rule:
\begin{equation}
    \vz_{n+1} = \mU \vz_n + \eps \left(\vz_n \had \vz_n\right) + \vc
    \label{eq:cy_main}
\end{equation}
where:
\begin{align}
    \vz_n &\in \C^k && \text{(complex state vector)} \\
    \mU &\in \C^{k \times k} && \text{(unitary matrix: } \mU^\dagger\mU = \mI\text{)} \\
    \eps &\in \C && \text{(nonlinearity strength)} \\
    \vc &\in \C^k && \text{(complex parameter vector)} \\
    \had && \text{(Hadamard/elementwise product)}
\end{align}
\end{definition}

\subsection{Component Interpretation}

\subsubsection{Unitary Evolution: $\mU\vz$}

The unitary matrix provides:
\begin{itemize}
    \item \textbf{Norm preservation}: $\norm{\mU\vz} = \norm{\vz}$
    \item \textbf{Eigenvalues on unit circle}: $\mu_i = e^{i\theta_i}$
    \item \textbf{Quantum analogy}: Similar to unitary evolution in quantum mechanics
\end{itemize}

\begin{proposition}
Any unitary matrix $\mU$ can be decomposed as:
\begin{equation}
    \mU = e^{i\mH}
\end{equation}
for Hermitian $\mH$.
\end{proposition}

\subsubsection{Nonlinear Term: $\eps(\vz \had \vz)$}

The Hadamard square provides:
\begin{equation}
    \vz \had \vz = (z_1^2, z_2^2, \ldots, z_k^2)^T
\end{equation}

This is the natural complex-quadratic extension:
\begin{itemize}
    \item For $k=1$, $\mU=1$, $\eps=1$: Reduces to $z \mapsto z^2 + c$ (Mandelbrot map)
    \item Couples magnitude and phase of each component
    \item Creates fractal boundary structure
\end{itemize}

% ----------------------------------------------------------------------------
\section{Alternative Formulations}
% ----------------------------------------------------------------------------

\subsection{Hermitian-Weighted Update}

\begin{definition}[Hermitian CY Dynamics]
An alternative with tanh nonlinearity:
\begin{equation}
    \vz_{n+1} = \mH\vz_n + \mB\tanh(\mU\vz_n) + \vc
\end{equation}
where $\mH = \mH^\dagger$ is Hermitian.
\end{definition}

Here $\tanh$ on complex arguments uses:
\begin{equation}
    \tanh(z) = \frac{e^z - e^{-z}}{e^z + e^{-z}}
\end{equation}

\subsection{Split Real-Imaginary Dynamics}

Writing $\vz = \vx + i\vy$:
\begin{align}
    \vx_{n+1} &= \mA\vx_n - \mC\vy_n + \mB\tanh(\mW\vx_n) + \Re(\vc) \\
    \vy_{n+1} &= \mC\vx_n + \mA\vy_n + \mB\tanh(\mW\vy_n) + \Im(\vc)
\end{align}

This makes real-imaginary coupling explicit and connects to the $2k$-dimensional real dynamics.

% ----------------------------------------------------------------------------
\section{Geometric Structure}
% ----------------------------------------------------------------------------

\subsection{Hermitian Inner Product}

\begin{definition}[Hermitian Inner Product]
On $\C^k$:
\begin{equation}
    \langle \vz, \vw \rangle = \vz^\dagger \vw = \sum_{j=1}^k \overline{z_j} w_j
\end{equation}
with norm $\norm{\vz} = \sqrt{\langle \vz, \vz \rangle}$.
\end{definition}

\subsection{Symplectic Structure}

\begin{definition}[Symplectic Form]
\begin{equation}
    \omega(\vz, \vw) = \Im\langle \vz, \vw \rangle
\end{equation}
\end{definition}

The unitary part preserves $\omega$, analogous to Hamiltonian mechanics.

% ----------------------------------------------------------------------------
\section{Jacobian and Stability}
% ----------------------------------------------------------------------------

\subsection{Complex Jacobian}

\begin{theorem}
For the CY dynamics \eqref{eq:cy_main}:
\begin{equation}
    \mJ(\vz) = \mU + 2\eps\diag(\vz)
\end{equation}
\end{theorem}

\subsection{Fixed Point Stability}

Fixed points $\zstar$ satisfy:
\begin{equation}
    \zstar = \mU\zstar + \eps(\zstar \had \zstar) + \vc
\end{equation}

Stability requires $|\mu_i| < 1$ for all eigenvalues of $\mJ(\zstar)$.

For small $\eps$:
\begin{equation}
    \mJ(\zstar) \approx \mU + O(\eps)
\end{equation}

Since unitary eigenvalues have $|\mu| = 1$, perturbations can push eigenvalues across the unit circle, causing bifurcations.

% ----------------------------------------------------------------------------
\section{Connection to Mandelbrot Dynamics}
% ----------------------------------------------------------------------------

\subsection{One-Dimensional Case}

Setting $k=1$, $\mU=1$, $\eps=1$:
\begin{equation}
    z_{n+1} = z_n^2 + c
\end{equation}

This is the Mandelbrot iteration.

\begin{definition}[Mandelbrot Set]
\begin{equation}
    \mathcal{M} = \{c \in \C : \sup_n |z_n| < \infty, z_0 = 0\}
\end{equation}
\end{definition}

\begin{definition}[Julia Set]
For fixed $c$, the filled Julia set:
\begin{equation}
    K_c = \{z_0 \in \C : \sup_n |z_n| < \infty\}
\end{equation}
The Julia set $J_c = \partial K_c$ is its boundary.
\end{definition}

\subsection{Higher-Dimensional Generalization}

The CY model generalizes Mandelbrot dynamics to $\C^k$:
\begin{itemize}
    \item Component-wise quadratic nonlinearity
    \item Unitary mixing between components
    \item Fractal structure in $2k$ real dimensions
\end{itemize}

% ----------------------------------------------------------------------------
\section{Lyapunov Analysis}
% ----------------------------------------------------------------------------

\subsection{Complex Lyapunov Exponent}

\begin{definition}
\begin{equation}
    \lyap = \lim_{n \to \infty} \frac{1}{n} \sum_{k=0}^{n-1} \log \norm{\mJ(\vz_k)}
\end{equation}
\end{definition}

\subsection{Lyapunov Spectrum}

For $\C^k$ dynamics (viewed as $\R^{2k}$):
\begin{itemize}
    \item $2k$ Lyapunov exponents
    \item Often appear in conjugate pairs due to complex structure
\end{itemize}

% ----------------------------------------------------------------------------
\section{Projections and Slices}
% ----------------------------------------------------------------------------

\subsection{Visualization Projections}

Since $\C^k$ is high-dimensional, we project to 2D:

\begin{definition}[Projection Methods]
\begin{align}
    P_{\text{Re-Im}}(\vz) &= (\Re(z_1), \Im(z_1)) \\
    P_{\text{components}}(\vz) &= (\Re(z_1), \Re(z_2)) \\
    P_{\text{stereo}}(\vz) &= \text{stereographic}(z_1/z_2)
\end{align}
\end{definition}

\subsection{Parameter Slices}

Fix all but one complex parameter:
\begin{equation}
    \vc(t) = \vc_0 + t \cdot \mathbf{e}_j, \quad t \in \C
\end{equation}

This produces 2D slices through parameter space, revealing Mandelbrot-like fractal structure.

% ----------------------------------------------------------------------------
\section{Physical Interpretation}
% ----------------------------------------------------------------------------

\subsection{Consciousness Modeling}

The complex structure provides:
\begin{itemize}
    \item \textbf{Amplitude}: Intensity or strength of mental state
    \item \textbf{Phase}: Quality or character of state
    \item \textbf{Interference}: Constructive/destructive combination of states
\end{itemize}

\subsection{Dimensional Interpretation}

For $k=3$ (inspired by 6D Calabi-Yau):
\begin{itemize}
    \item 3 complex dimensions = 6 real dimensions
    \item Rich dynamical structure
    \item Natural for modeling multi-aspect states
\end{itemize}

% ----------------------------------------------------------------------------
\section{Numerical Considerations}
% ----------------------------------------------------------------------------

\subsection{Escape Criterion}

\begin{definition}
Escape radius:
\begin{equation}
    R = \max(2, \norm{\vc})
\end{equation}
If $\norm{\vz_n} > R$, the orbit escapes to infinity.
\end{definition}

\subsection{Computational Parameters}

Typical settings:
\begin{itemize}
    \item Maximum iterations: $N_{\max} = 1000$
    \item Escape threshold: $R = 10$
    \item Convergence tolerance: $\delta = 10^{-8}$
\end{itemize}

% ----------------------------------------------------------------------------
\section{Summary}
% ----------------------------------------------------------------------------

The CY extension provides:
\begin{enumerate}
    \item Complex state space with Hermitian structure
    \item Unitary evolution preserving norms
    \item Natural connection to Mandelbrot/Julia dynamics
    \item Higher-dimensional generalization with rich structure
    \item Framework for the Possibility Manifold (Chapter 4)
\end{enumerate}
