% ============================================================================
% Chapter 7: Visualization and Interfaces
% ============================================================================

\chapter{Visualization and Interfaces}
\label{ch:visualization}

This chapter describes the tools and interfaces for exploring fractal dynamics across different platforms.

% ----------------------------------------------------------------------------
\section{Overview of Interfaces}
% ----------------------------------------------------------------------------

MindFractal Lab provides multiple interfaces:

\begin{table}[h]
\centering
\begin{tabular}{lll}
\toprule
\textbf{Interface} & \textbf{Use Case} & \textbf{Platform} \\
\midrule
Python API & Scripting, research & All \\
CLI & Terminal workflows & All \\
Kivy GUI & Interactive exploration & Android, Desktop \\
FastAPI Web & Browser-based access & Web \\
Jupyter & Notebook integration & All \\
Pyodide/Web & Client-side computation & Browser \\
\bottomrule
\end{tabular}
\caption{Available interfaces}
\end{table}

% ----------------------------------------------------------------------------
\section{Command Line Interface}
% ----------------------------------------------------------------------------

\subsection{Basic Commands}

\begin{verbatim}
# Simulate trajectory
mindfractal simulate --x0 0.5 0.5 --steps 1000

# Generate visualization
mindfractal visualize --mode orbit --output orbit.png

# Compute fractal map
mindfractal fractal --resolution 500 --output fractal.png

# Analyze stability
mindfractal analyze --lyapunov --bifurcation

# Tenth dimension tools
mindfractal possibility --explore
\end{verbatim}

\subsection{CLI Architecture}

\begin{itemize}
    \item Built with \texttt{argparse}
    \item Subcommand structure
    \item JSON/CSV output for scripting
    \item Matplotlib for visualizations
\end{itemize}

% ----------------------------------------------------------------------------
\section{Visualization Types}
% ----------------------------------------------------------------------------

\subsection{Phase Portraits}

Display trajectories in state space:
\begin{itemize}
    \item Single trajectory with color gradient
    \item Multiple trajectories (initial condition grid)
    \item Vector field overlay
    \item Fixed point markers
\end{itemize}

\subsection{Basin of Attraction}

Color-coded by attractor:
\begin{itemize}
    \item Fixed point → solid color
    \item Limit cycle → cycle-related color
    \item Chaos → gradient by Lyapunov
    \item Divergence → black/white
\end{itemize}

\subsection{Lyapunov Maps}

Parameter-space coloring:
\begin{itemize}
    \item Blue: $\lyap < 0$ (stable)
    \item White: $\lyap \approx 0$ (neutral)
    \item Red: $\lyap > 0$ (chaotic)
\end{itemize}

\subsection{Fractal Slices}

Mandelbrot/Julia-style:
\begin{itemize}
    \item Escape-time coloring
    \item Smooth coloring with potential function
    \item Orbit trap methods
\end{itemize}

\subsection{3D Visualizations}

\begin{itemize}
    \item 3D trajectory plots
    \item Attractor projections
    \item Basin slice stacks
    \item Interactive rotation
\end{itemize}

% ----------------------------------------------------------------------------
\section{Kivy GUI}
% ----------------------------------------------------------------------------

\subsection{Features}

\begin{itemize}
    \item Real-time parameter sliders
    \item Touch-based navigation
    \item Live trajectory animation
    \item Save/load configurations
\end{itemize}

\subsection{Architecture}

\begin{verbatim}
mindfractal_app.py
├── MainScreen
│   ├── ParameterPanel (sliders for A, B, W, c)
│   ├── VisualizationCanvas (matplotlib/kivy)
│   └── ControlButtons (simulate, save, reset)
├── SettingsScreen
└── HelpScreen
\end{verbatim}

\subsection{Mobile Optimization}

\begin{itemize}
    \item Reduced resolution for performance
    \item Touch gestures for zoom/pan
    \item Efficient NumPy computations
\end{itemize}

% ----------------------------------------------------------------------------
\section{FastAPI Web Application}
% ----------------------------------------------------------------------------

\subsection{Endpoints}

\begin{verbatim}
GET  /api/simulate
     ?x0=0.5,0.5&steps=1000
     Returns: JSON trajectory

POST /api/visualize
     Body: {config, mode, resolution}
     Returns: PNG image

GET  /api/lyapunov
     ?x0=...&c1=...&c2=...
     Returns: Lyapunov exponent

POST /api/fractal
     Body: {bounds, resolution}
     Returns: PNG fractal map
\end{verbatim}

\subsection{Frontend}

\begin{itemize}
    \item HTML templates with Jinja2
    \item JavaScript for interactivity
    \item Canvas for visualization
    \item Form-based parameter input
\end{itemize}

% ----------------------------------------------------------------------------
\section{Web Interactive (Pyodide)}
% ----------------------------------------------------------------------------

\subsection{Concept}

Run Python in the browser via WebAssembly:
\begin{itemize}
    \item No server computation required
    \item Full NumPy/Matplotlib support
    \item Client-side fractal generation
\end{itemize}

\subsection{Architecture}

\begin{verbatim}
docs/site/interactive/
├── index.html
├── js/
│   ├── pyodide_bootstrap.js
│   ├── fractal_viewer.js
│   └── cy_slice_viewer.js
└── py/
    ├── fractal_core.py
    ├── cy_core.py
    └── possibility_core.py
\end{verbatim}

\subsection{Implementation}

JavaScript loads Pyodide and Python modules:
\begin{verbatim}
// Load Pyodide
const pyodide = await loadPyodide();
await pyodide.loadPackage(['numpy', 'matplotlib']);

// Load custom modules
await pyodide.runPython(fractal_core_code);

// Compute and display
const result = pyodide.runPython(`
    compute_fractal(c1=${c1}, c2=${c2}, res=${res})
`);
displayImage(result);
\end{verbatim}

% ----------------------------------------------------------------------------
\section{Jupyter Integration}
% ----------------------------------------------------------------------------

\subsection{Interactive Widgets}

\begin{verbatim}
import ipywidgets as widgets
from mindfractal import FractalDynamicsModel, plot_orbit

@widgets.interact(
    c1=(-2, 2, 0.1),
    c2=(-2, 2, 0.1)
)
def explore(c1, c2):
    model = FractalDynamicsModel(c=[c1, c2])
    plot_orbit(model, [0.5, 0.5])
\end{verbatim}

\subsection{Rich Output}

\begin{itemize}
    \item Inline matplotlib figures
    \item Animation playback
    \item Interactive pan/zoom
    \item LaTeX equation rendering
\end{itemize}

% ----------------------------------------------------------------------------
\section{Visualization Algorithms}
% ----------------------------------------------------------------------------

\subsection{Progressive Rendering}

\begin{enumerate}
    \item Start at low resolution (64$\times$64)
    \item Display immediately
    \item Progressively refine (128, 256, 512, ...)
    \item Allow interaction during computation
\end{enumerate}

\subsection{Coloring Schemes}

\begin{definition}[Escape Time]
\begin{equation}
    \text{color} = \text{palette}(n + 1 - \log_2\log_2|z_n|)
\end{equation}
\end{definition}

\begin{definition}[Lyapunov Coloring]
\begin{equation}
    \text{color} = \begin{cases}
        \text{blue}(-\lyap) & \lyap < 0 \\
        \text{white} & \lyap \approx 0 \\
        \text{red}(\lyap) & \lyap > 0
    \end{cases}
\end{equation}
\end{definition}

\subsection{Animation}

\begin{itemize}
    \item Trajectory animation (growing path)
    \item Parameter sweep (varying $\vc$)
    \item Bifurcation animation (zoom into structure)
    \item Rotation animation (3D attractors)
\end{itemize}

% ----------------------------------------------------------------------------
\section{Performance Optimization}
% ----------------------------------------------------------------------------

\subsection{NumPy Vectorization}

\begin{verbatim}
# Vectorized basin computation
grid = np.meshgrid(x_range, y_range)
z = grid[0] + 1j * grid[1]

for _ in range(max_iter):
    z = A @ z + B @ np.tanh(W @ z) + c
    escaped = np.abs(z) > R
    # ... classification
\end{verbatim}

\subsection{Parallel Computation}

\begin{itemize}
    \item \texttt{multiprocessing} for CPU parallelism
    \item Row-based partitioning for basin maps
    \item Per-pixel independence for fractals
\end{itemize}

\subsection{C++ Backend}

\begin{itemize}
    \item pybind11 bindings
    \item 10-100x speedup for iteration loops
    \item Optional: GPU via CUDA/OpenCL
\end{itemize}

% ----------------------------------------------------------------------------
\section{Export and Sharing}
% ----------------------------------------------------------------------------

\subsection{Image Formats}

\begin{itemize}
    \item PNG: Lossless, good for fractals
    \item SVG: Vector graphics for diagrams
    \item GIF: Animations
    \item MP4: Video animations
\end{itemize}

\subsection{Data Formats}

\begin{itemize}
    \item JSON: Configuration and trajectories
    \item CSV: Tabular data
    \item NPZ: NumPy arrays (compact)
    \item HDF5: Large datasets
\end{itemize}

\subsection{Interactive Sharing}

\begin{itemize}
    \item GitHub Pages deployment
    \item Observable notebooks
    \item Colab notebooks
    \item Streamlit apps
\end{itemize}

% ----------------------------------------------------------------------------
\section{Future Interfaces}
% ----------------------------------------------------------------------------

\subsection{VR/AR}

\begin{itemize}
    \item 3D attractor immersion
    \item Gesture-based parameter control
    \item Spatial navigation through $\Pcal$
\end{itemize}

\subsection{WebXR}

\begin{itemize}
    \item Browser-based VR/AR
    \item Three.js visualization
    \item Cross-platform compatibility
\end{itemize}

\subsection{Haptic Feedback}

\begin{itemize}
    \item Vibration patterns for stability regions
    \item Force feedback near bifurcations
    \item Tactile exploration of basins
\end{itemize}

% ----------------------------------------------------------------------------
\section{Summary}
% ----------------------------------------------------------------------------

The visualization system provides:
\begin{enumerate}
    \item Multiple interfaces for different use cases
    \item Rich visualization types (phase, basin, fractal)
    \item Cross-platform support (desktop, mobile, web)
    \item Interactive exploration with real-time feedback
    \item Export and sharing capabilities
    \item Performance optimization strategies
\end{enumerate}
