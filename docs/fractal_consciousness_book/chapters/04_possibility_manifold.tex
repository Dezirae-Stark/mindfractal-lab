% ============================================================================
% Chapter 4: The Possibility Manifold
% ============================================================================

\chapter{The Possibility Manifold}
\label{ch:possibility_manifold}

This chapter formalizes the Possibility Manifold $\Pcal$---the space of all system configurations with bounded dynamics.

% ----------------------------------------------------------------------------
\section{Definition and Structure}
% ----------------------------------------------------------------------------

\subsection{Formal Definition}

\begin{definition}[Possibility Manifold]
Let $\Fcal = \{F_\alpha\}$ be a family of update rules. The Possibility Manifold is:
\begin{equation}
    \Pcal = \left\{ (\vz_0, \vc, F) \in \C^n \times \C^n \times \Fcal : \orbit(\vz_0, \vc, F) \text{ is bounded} \right\}
\end{equation}
where:
\begin{equation}
    \orbit(\vz_0, \vc, F) = \{\vz_0, F(\vz_0; \vc), F^2(\vz_0; \vc), \ldots\}
\end{equation}
\end{definition}

\subsection{Component Spaces}

\begin{definition}[Initial Condition Space]
For fixed $(\vc, F)$:
\begin{equation}
    \mathcal{Z}_0(\vc, F) = \{\vz_0 : (\vz_0, \vc, F) \in \Pcal\}
\end{equation}
This generalizes the filled Julia set.
\end{definition}

\begin{definition}[Parameter Space]
For fixed $(\vz_0, F)$:
\begin{equation}
    \mathcal{C}(\vz_0, F) = \{\vc : (\vz_0, \vc, F) \in \Pcal\}
\end{equation}
This generalizes the Mandelbrot set.
\end{definition}

\subsection{Update Rule Family}

\begin{definition}[Rule Family $\Fcal$]
The standard family includes:
\begin{align}
    F_{\tanh}: \quad &\vz_{n+1} = \mA\vz_n + \mB\tanh(\mW\vz_n) + \vc \\
    F_{\sigma}: \quad &\vz_{n+1} = \mA\vz_n + \mB\sigma(\mW\vz_n) + \vc \\
    F_{3D}: \quad &\vz_{n+1} = \mA\vz_n + \mB\tanh(\mW\vz_n) + \vc \quad (\vz \in \C^3) \\
    F_{\text{CY}}: \quad &\vz_{n+1} = \mU\vz_n + \eps(\vz_n \had \vz_n) + \vc
\end{align}
\end{definition}

% ----------------------------------------------------------------------------
\section{Topology}
% ----------------------------------------------------------------------------

\subsection{Product Topology}

$\Pcal$ inherits topology from:
\begin{equation}
    \C^n \times \C^n \times \Fcal
\end{equation}

For discrete $\Fcal$: product of Euclidean and discrete topologies.

\subsection{Boundary Structure}

\begin{definition}[Possibility Boundary]
\begin{equation}
    \partial\Pcal = \overline{\Pcal} \setminus \Pcal^\circ
\end{equation}
Points on $\partial\Pcal$ are \emph{critical}---small perturbations cause escape.
\end{definition}

\begin{theorem}[Fractal Boundary]
For polynomial/transcendental dynamics, $\partial\Pcal$ typically has fractal structure with:
\begin{equation}
    \dim_H(\partial\Pcal) > \dim_{\text{top}}(\partial\Pcal)
\end{equation}
\end{theorem}

% ----------------------------------------------------------------------------
\section{Metrics on $\Pcal$}
% ----------------------------------------------------------------------------

\subsection{Weighted Distance}

\begin{definition}[Possibility Distance]
\begin{equation}
    \dP(p_1, p_2) = \sqrt{w_1\norm{\vz_{0,1} - \vz_{0,2}}^2 + w_2\norm{\vc_1 - \vc_2}^2 + w_3 d_\Fcal(F_1, F_2)^2}
\end{equation}
with weights $w_1, w_2, w_3 \geq 0$.
\end{definition}

\subsection{Dynamical Distance}

\begin{definition}[Orbit Distance]
\begin{equation}
    d_{\text{dyn}}(p_1, p_2) = \frac{1}{N}\sum_{k=0}^{N-1} \norm{\vz_k^{(1)} - \vz_k^{(2)}}
\end{equation}
\end{definition}

This captures the idea that similar configurations produce similar dynamics.

% ----------------------------------------------------------------------------
\section{Stability Regions}
% ----------------------------------------------------------------------------

\subsection{Lyapunov Classification}

\begin{definition}[Stability Partition]
\begin{align}
    \Pcal_{\text{stable}} &= \{p : \lyap(p) < -\delta\} \\
    \Pcal_{\text{chaotic}} &= \{p : \lyap(p) > \delta\} \\
    \Pcal_{\text{boundary}} &= \{p : |\lyap(p)| \leq \delta\}
\end{align}
\end{definition}

\subsection{Attractor Classification}

\begin{definition}[Attractor Regions]
\begin{align}
    \Pcal_{\text{fixed}} &= \{p : \orbit(p) \to \text{fixed point}\} \\
    \Pcal_{\text{periodic}} &= \{p : \orbit(p) \to \text{limit cycle}\} \\
    \Pcal_{\text{strange}} &= \{p : \orbit(p) \to \text{strange attractor}\}
\end{align}
\end{definition}

% ----------------------------------------------------------------------------
\section{Timelines and Paths}
% ----------------------------------------------------------------------------

\subsection{Timeline Definition}

\begin{definition}[Timeline]
A timeline is a curve $\gamma: [0,1] \to \Pcal$:
\begin{equation}
    \gamma(t) = (\vz_0(t), \vc(t), F(t))
\end{equation}
\end{definition}

\subsection{Linear Timeline}

\begin{definition}[Linear Interpolation]
\begin{equation}
    \gamma(t) = (1-t)p_1 + tp_2
\end{equation}
connecting $p_1$ and $p_2$.
\end{definition}

\subsection{Geodesics}

Shortest paths in $(\Pcal, \dP)$:
\begin{equation}
    \gamma^* = \arg\min_\gamma \int_0^1 \norm{\dot\gamma(t)} dt
\end{equation}

% ----------------------------------------------------------------------------
\section{Bifurcations and Branching}
% ----------------------------------------------------------------------------

\subsection{Bifurcation Locus}

\begin{definition}[Bifurcation Points]
\begin{equation}
    \mathcal{B} = \{p \in \Pcal : \text{qualitative dynamics change at } p\}
\end{equation}
\end{definition}

Types include:
\begin{itemize}
    \item Saddle-node bifurcation
    \item Period-doubling cascade
    \item Hopf bifurcation
    \item Crisis (sudden attractor change)
\end{itemize}

\subsection{Branching Structure}

\begin{definition}[Branch Point]
A point where multiple attractors emerge:
\begin{equation}
    p^* \in \bigcap_{i=1}^k \overline{\Bcal(\Acal_i)}
\end{equation}
\end{definition}

This formalizes ``branching realities'' as passage through bifurcation points.

% ----------------------------------------------------------------------------
\section{Sampling and Exploration}
% ----------------------------------------------------------------------------

\subsection{Random Sampling}

\begin{algorithm}[H]
\caption{Sample from $\Pcal$}
\begin{algorithmic}[1]
\REPEAT
    \STATE $\vz_0 \sim \text{Uniform}(B_R)$
    \STATE $\vc \sim \text{Uniform}(B_R)$
    \STATE $F \sim \text{Uniform}(\Fcal)$
    \STATE Test if $(\vz_0, \vc, F) \in \Pcal$
\UNTIL{$(\vz_0, \vc, F) \in \Pcal$}
\RETURN $(\vz_0, \vc, F)$
\end{algorithmic}
\end{algorithm}

\subsection{Slice Visualization}

\begin{definition}[2D Slice]
\begin{equation}
    S = \{p(\alpha, \beta) : \alpha, \beta \in [-R, R]\}
\end{equation}
parameterizing a 2D plane through $\Pcal$.
\end{definition}

% ----------------------------------------------------------------------------
\section{Dimension of $\Pcal$}
% ----------------------------------------------------------------------------

\begin{proposition}
For $n$-dimensional complex dynamics with $|\Fcal| = m$ rules:
\begin{equation}
    \dim(\Pcal) \leq 4n + \log_2 m
\end{equation}
\end{proposition}

Example: $n=3$, $m=4$:
\begin{equation}
    \dim(\Pcal) \leq 4(3) + 2 = 14
\end{equation}

% ----------------------------------------------------------------------------
\section{Connection to Consciousness}
% ----------------------------------------------------------------------------

\subsection{Interpretation}

\begin{itemize}
    \item $\vz$: Current mental state
    \item $\vc$: Personality/context parameters
    \item $F$: Cognitive processing style
    \item $\Pcal$: All viable mental configurations
\end{itemize}

\subsection{Applications}

\begin{itemize}
    \item \textbf{State assessment}: Locate current $p \in \Pcal$
    \item \textbf{Goal setting}: Identify target $p^* \in \Pcal_{\text{stable}}$
    \item \textbf{Intervention}: Perturb $\vc$ to shift dynamics
\end{itemize}

% ----------------------------------------------------------------------------
\section{Summary}
% ----------------------------------------------------------------------------

The Possibility Manifold provides:
\begin{enumerate}
    \item Rigorous definition of ``all possibilities''
    \item Natural metric and topological structure
    \item Classification by stability and attractor type
    \item Framework for timelines and branching
    \item Foundation for the tenth dimension metaphor (Chapter 5)
\end{enumerate}
